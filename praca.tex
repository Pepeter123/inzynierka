%\documentclass[a4paper,12pt]{book}
\documentclass[a4paper,12pt,openany]{book}
\input{Pakiety/pakiety.tex}
\input{Ustawienia/ustawienia.tex}


%==========================================================================================
% Dane na temat pracy do wype?nienia
%==========================================================================================
\author{Piotr Noga}
\kierunek{Informatyka \\ Specjalno��: Sieciowe Systemy Informatyczne \\ Studia stacjonarne}
\grupa{43INF-SSI}		% np. 101AiRDz
\title{System do wykrywania i rozpoznawania obiekt�w na obrazie z kamery samochodowej z wykorzystaniem symulatora carla}
\tytulAngielski{Title of diploma}
\uczelnia{Uniwersytet Zielonog�rski}
\wydzial{Wydzia� Nauk In�ynieryjno-Technicznych \\ Instytut Automatyki, Elektroniki i Elektrotechniki
}
\praca{Praca in�ynierska} % pozostawi? w?a?ciwe
\promotor{Dr hab. In�. Marek Kowal, prof. UZ}
\konsultant{stopie� naukowy, imi� i~nazwisko konsultanta} 	% nie usuwa? w~przypadku braku lecz pozostawi? puste czyli: \konsultant{}
\konsultant{}
\miasto{Zielona G�ra}
\miesiac{kwiecie�}	% miesi?c z?o?enia pracy
\rok{2025}
\dzien{} 			% dzie? podpisania o?wiadczenia (cyfrowo) np. 10
\mm{mm} 				% miesiac podpisania o?wiadczenia (cyfrowo) np. 01 (stycze?)

%==========================================================================================
% Uwaga! polecenie \myemptypage dodaje pust? stron? do tre?ci. Praca oddana do dziekanatu powinna by? zbudowana zgodnie z~szablonem znajduj?cym si? na stronie WEIT (z jedn? stron? pust? nast?puj?c? po karcie pracy), jednak je?eli student planuje wydruk dla siebie, zalecane jest zast?pienie polece? \newpage poleceniem \myemptypage. Skutkuje to utworzeniem bardziej przejrzystego uk?adu zgodnego z~zaleceniami pisania ksi??ek, czyli rozpoczynanie nowych treci na prawej stronie.
%==========================================================================================


%==========================================================================================
% Dokument glowny
%==========================================================================================
\begin{document}
\pagenumbering{roman}
%=========================================================================================
% Strona tytu?owa
%=========================================================================================
\thispagestyle{empty}
\stronatytulowa
\thispagestyle{empty}
\pustastrona
\newpage
%=========================================================================================
% O?wiadczenie o~wsp�?realizacji pracy - w~przypadku braku zakomentowa? lub usun?? sekcj?
%========================================================================================
%\autorDwa{Imi? i~nazwisko wsp�autora}
%\wspolrealizacjaTrescJeden{wykaz czynno?ci wykonanych przez t? osob?}
%\wspolrealizacjaTrescDwa{wykaz czynno?ci wykonanych przez t? osob?}
%\wspolrealizacja
%\newpage

%\thispagestyle{empty}
%\pustastrona
%\newpage

%========================================================================================
% Streszczenie
%========================================================================================
\normalsize

\subsection*{Streszczenie}

Celem niniejszej pracy by�o opracowanie systemu wykrywania i rozpoznawania obiekt�w na obrazach pochodz�cych z kamery samochodowej, z zastosowaniem algorytmu YOLOv4 oraz symulatora jazdy samochodowej CARLA. Projekt zak�ada� integracj� detektora obiekt�w z symulowanym �rodowiskiem oraz ocen� jego skuteczno�ci w r�nych warunkach drogowych i pogodowych.

W pierwszej cz�ci pracy przedstawiono charakterystyk� �rodowiska CARLA oraz przegl�d metod wykrywania obiekt�w w obrazach. Nast�pnie zaprojektowano i zaimplementowano system detekcji, kt�ry umo�liwia identyfikacj� pojazd�w, pieszych i znak�w drogowych w czasie rzeczywistym. System zosta� przetestowany w symulowanych warunkach, a uzyskane wyniki wykaza�y wysok� skuteczno�� detekcji oraz stabilno�� dzia�ania.

Na podstawie przeprowadzonych bada� stwierdzono, �e opracowany system spe�nia za�o�one cele i mo�e stanowi� podstaw� do dalszych prac nad percepcj� �rodowiska w pojazdach autonomicznych.


% d�u�sze
% W niniejszej pracy in�ynierskiej przedstawiono projekt oraz implementacj� systemu wykrywania i rozpoznawania obiekt�w w �rodowisku symulacyjnym CARLA z wykorzystaniem algorytmu YOLOv4. Celem pracy by�o stworzenie kompletnego rozwi�zania umo�liwiaj�cego analiz� otoczenia pojazdu w czasie rzeczywistym na podstawie obrazu z kamery samochodowej, w kontek�cie zastosowania w systemach wspomagania kierowcy oraz pojazdach autonomicznych.      Praca zosta�a rozpocz�ta od zapoznania si� z architektur� i funkcjonalno�ciami symulatora CARLA, kt�ry dostarcza realistyczne warunki symulacyjne dla ruchu drogowego, uwzgl�dniaj�c r�ne warunki atmosferyczne, pory dnia, infrastruktur� miejsk� i pozamiejsk�. W kolejnych etapach przeprowadzono analiz� wsp�czesnych metod detekcji obiekt�w, koncentruj�c si� na rodzinie algorytm�w YOLO (You Only Look Once), ze szczeg�lnym uwzgl�dnieniem YOLOv4, kt�ry oferuje wysok� precyzj� oraz efektywno�� dzia�ania przy detekcji wielu klas obiekt�w.         G��wna cz�� pracy skupia si� na integracji modelu detekcji YOLOv4 z kodem �r�d�owym CARLA, a w szczeg�lno�ci z modu�em manual_control.py, odpowiedzialnym za sterowanie i wizualizacj� symulacji. Zaimplementowany system umo�liwia wykrywanie pojazd�w, pieszych, znak�w drogowych i innych istotnych element�w infrastruktury drogowej. Obiekty s� oznaczane w czasie rzeczywistym na ekranie symulatora za pomoc� prostok�t�w oraz etykiet opisuj�cych klas� rozpoznanego obiektu. W trakcie implementacji uwzgl�dniono wykorzystanie GPU, optymalizacj� przetwarzania obrazu oraz odpowiednie przekszta�cenia i normalizacj� danych wej�ciowych.       Zwie�czeniem pracy by�a seria test�w przeprowadzonych w r�nych scenariuszach symulacyjnych. Uzyskane wyniki potwierdzi�y skuteczno�� dzia�ania systemu � wykrywane by�y wszystkie obiekty istotne z punktu widzenia system�w bezpiecze�stwa i autonomii pojazdu. Opracowane oprogramowanie mo�e by� z powodzeniem wykorzystane jako narz�dzie badawcze, a tak�e jako prototyp systemu do zastosowania w realnych warunkach.

% stare
% W pracy opisane zosta�y technologie jak i kroki niezb�dne do stworzenia systemu do wykrywania i rozpoznawania obiekt�w na podstawie obraz�w pochodz�cych z kamery samochodowej z wykorzystaniem symulatora CARLA. W tym celu u�yty zosta� system wykrywania obiekt�w w czasie rzeczywistym YOLOv4 na platformie sieci neuronowych Darknet, napisanym w j�zyku Python.    Pokazano w jaki spos�b zainstalowa� symulator CARLA w dystrybucji systemu Linux a mianowicie Ubuntu. Przedstawiony zosta� proces instalacji zale�no�ci, pobierania �r�de� CARLA, kompilacji projektu, uruchamiania symulatora oraz instalacji niezb�dnych pakiet�w Pythona. Dzi�ki tym krokom zosta�o przygotowane �rodowisko, aby zacz�� korzysta� z CARLA i eksperymentowa� z symulacjami pojazd�w autonomicznych.          Skrypt \texttt{manual\_control.py} w symulatorze CARLA pozwala na interakcj� z pojazdem w spos�b manualny, wykorzystuj�c standardowe urz�dzenia wej�ciowe, takie jak klawiatura. Jego struktura opiera si� na kilku kluczowych etapach: konfiguracji klienta, obs�udze wej�cia u�ytkownika, g��wnej p�tli gry oraz czyszczeniu zasob�w po zako�czeniu symulacji. Dzi�ki odpowiedniemu zarz�dzaniu wej�ciem i kontrol� pojazdu, mo�liwe jest uzyskanie realistycznego i intuicyjnego do�wiadczenia symulacyjnego.          Zaimplementowane oprogramowanie demonstruje kompleksowe podej�cie do integracji algorytm�w sztucznej inteligencji z symulatorem jazdy autonomicznej. System umo�liwia dynamiczn� analiz� otoczenia pojazdu, interaktywne testowanie scenariuszy oraz wizualn� walidacj� wynik�w, co czyni go niezwykle cennym narz�dziem in�ynierskim i badawczym.



% Mateusza
% Przetwarzanie danych oraz tworzenie zapyta� przy u�yciu sk�adni SQL, umo�liwi� Hive Apache jako silnik hurtowni danych. G��wnym elementem ka�dej hurtowni danych jest analiza du�ych zbior�w informacji, do kt�rej wykorzystane zosta�o oprogramowanie Pentaho zapewniaj�ce analiz� biznesow�. Za ?rodowisko pracy wykorzystano Ubuntu, czyli jedna z najpopularniejszych dystrybucji Linuxa na ?wiecie.
%



\vspace{1cm}
\noindent\textbf{S�owa kluczowe:} symulator CARLA, YOLO4,  Darknet, Python, Ubuntu, detekcja obiekt�w, rozpoznawanie obraz�w, sztuczna inteligencja
%Mateusza
%hurtownia danych, Apache Hadoop, Apache Hive, Pentaho, Ubuntu.

\newpage
%\myemptypage

%========================================================================================
% Spis tresci, spis tabel i~rysunk�w
%========================================================================================
%spis tresci
	\tableofcontents
	\newpage
	%\myemptypage
%spis rysunk�w
 	\listoffigures
	\newpage
	%\myemptypage
%spis tabel
	\listoftables
	\newpage
%spis listing�w
	{\raggedbottom
		\lstlistoflistings\raggedbottom}
	%\myemptypage

%========================================================================================
% Licznik Stron
%========================================================================================
\newcounter{licznikStron}
\setcounter{licznikStron}{\value{page}}
\setcounter{licznikStron}{1}
\pagenumbering{arabic}
\setcounter{page}{\value{licznikStron}}

%========================================================================================
% Tresc
%========================================================================================
\chapter{Wst�p}

%======================================================================================================
\section{Wprowadzenie}

Wsp�czesna motoryzacja dynamicznie ewoluuje, a jednym z kluczowych obszar�w jej rozwoju s� systemy rozpoznawania i wykrywania obiekt�w. Technologie te, wykorzystuj�ce sztuczn� inteligencj� i uczenie maszynowe, pozwalaj� na analiz� otoczenia w czasie rzeczywistym, co znajduje zastosowanie zar�wno w systemach wspomagania kierowcy (ADAS), jak i w pojazdach autonomicznych. Dzi�ki nim mo�liwe jest m.in. rozpoznawanie znak�w drogowych, wykrywanie przeszk�d czy monitorowanie martwego pola, co przek�ada si� na zwi�kszenie bezpiecze�stwa na drogach.

Jednym z g��wnych wyzwa� w rozwijaniu tych technologii jest ich testowanie w realistycznych warunkach. Tradycyjne metody, takie jak rzeczywiste jazdy testowe, s� kosztowne i czasoch�onne, a dodatkowo mog� wi�za� si� z ryzykiem wypadk�w. Dlatego coraz cz�ciej wykorzystuje si� symulacje komputerowe, kt�re pozwalaj� na przeprowadzanie eksperyment�w w kontrolowanych warunkach. Jednym z najbardziej zaawansowanych narz�dzi w tej dziedzinie jest symulator CARLA (Car Learning to Act) � otwarto�r�d�owe �rodowisko stworzone z my�l� o badaniach nad autonomiczn� jazd� oraz systemami percepcji wizualnej pojazd�w~\cite{carla}.

W niniejszej pracy zostan� om�wione kluczowe aspekty wykrywania i rozpoznawania obiekt�w na podstawie obrazu z kamer samochodowych, a tak�e mo�liwo�ci wykorzystania symulatora CARLA do bada� w tym zakresie.

Przyk�ady zastosowa� system�w rozpoznawania obrazu w motoryzacji:
\begin{itemize}
	\item \textbf{Systemy wspomagania kierowcy (ADAS)}
	\begin{itemize}
		\item \textbf{Rozpoznawanie znak�w drogowych} � pozwala na identyfikacj� znak�w takich jak ograniczenia pr�dko�ci czy zakazy wyprzedzania, pomagaj�c kierowcom w przestrzeganiu przepis�w.
		\item \textbf{Ostrzeganie o niezamierzonej zmianie pasa ruchu} � system analizuje po�o�enie pojazdu i informuje kierowc� o niekontrolowanym opuszczeniu pasa.
		\item \textbf{Adaptacyjny tempomat} � automatycznie dostosowuje pr�dko�� pojazdu do warunk�w na drodze, utrzymuj�c bezpieczn� odleg�o�� od innych pojazd�w.
	\end{itemize}
\end{itemize}

\begin{itemize}
	\item \textbf{Systemy bezpiecze�stwa}
	\begin{itemize}
		\item \textbf{Automatyczne hamowanie awaryjne} � wykrywa potencjalne kolizje i inicjuje hamowanie, aby zminimalizowa� ryzyko wypadku.
		\item \textbf{Monitorowanie martwego pola} � ostrzega kierowc� o pojazdach znajduj�cych si� w obszarach niewidocznych w lusterkach.
	\end{itemize}
\end{itemize}

\begin{itemize}
	\item \textbf{Autonomiczne pojazdy i analiza otoczenia}
	\begin{itemize}
		\item \textbf{Identyfikacja przeszk�d i u�ytkownik�w drogi} � pozwala na wykrywanie pieszych, rowerzyst�w oraz innych pojazd�w, umo�liwiaj�c pojazdom autonomicznym bezpieczne poruszanie si� po drogach.
		\item \textbf{Predykcja zachowa� innych uczestnik�w ruchu} � analiza sygna��w drogowych, ruchu pieszych i pojazd�w umo�liwia przewidywanie ich manewr�w i odpowiednie reagowanie.
	\end{itemize}
\end{itemize}

\begin{itemize}
	\item \textbf{Systemy wspomagaj�ce parkowanie}
	\begin{itemize}
		\item \textbf{Automatyczne parkowanie} � pojazd mo�e samodzielnie wykona� manewry parkowania, korzystaj�c z kamer i czujnik�w.
		\item \textbf{Widok 360�} � system kamer rozmieszczonych wok� pojazdu u�atwia manewrowanie w ciasnych przestrzeniach.
	\end{itemize}
\end{itemize}

\begin{itemize}
	\item \textbf{Monitorowanie stanu kierowcy}
	\begin{itemize}
		\item \textbf{Systemy wykrywaj�ce zm�czenie} � analizuj� ruchy kierowcy, takie jak cz�stotliwo�� mrugania czy pozycja g�owy, ostrzegaj�c w przypadku wykrycia oznak zm�czenia.
	\end{itemize}
\end{itemize}
% 
%======================================================================================================

\section{Cel i zakres pracy}

Celem pracy jest opracowanie systemu wykrywania i rozpoznawania obiekt�w takich jak znaki drogowe, samochody oraz ludzie na obrazach pochodz�cych z kamery samochodowej. Projekt zostanie zaimplementowany z wykorzystaniem j�zyka Python 2, przy pomocy darmowego �rodowiska CARLA umo�liwiaj�cego symulacj� jazdy samochodem w r�nych warunkach drogowych.

Zakres pracy obejmowa�:
\begin{itemize}
	\item zapoznanie si� ze �rodowiskiem do symulacji jazdy samochodem CARLA,
	\item przegl�d metod wykrywania obiekt�w na obrazach z kamery samochodowej,
	\item zaprojektowanie i wdro�enie systemu wykrywania obiekt�w w �rodowisku CARLA,
	\item przeprowadzenie test�w weryfikuj�cych skuteczno�� zaprojektowanego systemu,
	\item sformu�owanie wniosk�w.
\end{itemize}


\section{Struktura pracy}

Niniejsza praca sk�ada si� z pi�ciu rozdzia��w, kt�re prowadz� od wprowadzenia teoretycznego, przez opis implementacji, a� do prezentacji wynik�w bada� eksperymentalnych i wniosk�w ko�cowych.

W \textbf{rozdziale 1} przedstawiono t�o problemu, motywacj� podj�cia tematyki detekcji obiekt�w na potrzeby pojazd�w autonomicznych, a tak�e zdefiniowano cel i zakres pracy oraz om�wiono jej struktur�. Zwr�cono uwag� na znaczenie symulacji komputerowych w procesie testowania algorytm�w percepcji.

\textbf{Rozdzia� 2} po�wi�cony jest szczeg�owemu opisowi symulatora CARLA. Zaprezentowano w nim architektur� systemu klient�serwer, dost�pne sensory, spos�b modelowania �wiata oraz mo�liwo�ci konfiguracji warunk�w �rodowiskowych. Om�wiono r�wnie� spos�b instalacji symulatora w systemie Ubuntu oraz struktur� skryptu \texttt{manualcontrol.py}, kt�ry stanowi podstaw� integracji z zewn�trznymi algorytmami detekcji.

W \textbf{rozdziale 3} opisano system rozpoznawania obrazu zastosowany w pracy. Przedstawiono wybrane architektury detekcji obiekt�w, ze szczeg�lnym uwzgl�dnieniem algorytmu YOLOv4, jego budowy i technik treningowych. Nast�pnie zaprezentowano proces instalacji i uruchomienia detektora oraz spos�b jego integracji z symulatorem CARLA. Na ko�cu rozdzia�u przedstawiono schemat ewaluacji offline, wykorzystywany do obliczania miary IoU na podstawie danych referencyjnych generowanych przez symulator.

\textbf{Rozdzia� 4} zawiera opis procesu testowania zintegrowanego systemu oraz wyniki bada� eksperymentalnych. Om�wiono konfiguracj� scenariuszy jazdy, spos�b rejestracji danych oraz wybrane funkcjonalno�ci oprogramowania. Zaprezentowano wyniki eksperymentu on-line, dotycz�ce wydajno�ci systemu mierzonej liczb� klatek na sekund�, oraz eksperymentu offline, w kt�rym oceniano poprawno�� detekcji za pomoc� miary IoU w r�nych warunkach pogodowych i o�wietleniowych.

W \textbf{rozdziale 5} przedstawiono dyskusj� uzyskanych rezultat�w oraz wnioski ko�cowe. Podsumowano najwa�niejsze obserwacje dotycz�ce jako�ci i wydajno�ci detekcji obiekt�w w symulatorze CARLA z wykorzystaniem algorytmu YOLOv4, a tak�e wskazano mo�liwe kierunki dalszych prac, w szczeg�lno�ci zwi�zane z rozszerzeniem zbioru danych, zastosowaniem innych architektur detekcji oraz integracj� dodatkowych sensor�w.

\chapter{Symulator CARLA}

CARLA (Car Learning to Act) jest otwartym symulatorem jazdy miejskiej, przeznaczonym do wspierania szkole�, trening�w, tworzenia prototyp�w oraz walidacji autonomicznych system�w jazdy zar�wno na poziomie percepcji, jak i kontroli. System stanowi otwart� platform� stworzon� przez specjalistyczny zesp� grafik�w i in�ynier�w, obejmuj�c� uk�ady urbanistyczne, modele pojazd�w, budynki, pieszych oraz znaki drogowe. Symulacja umo�liwia elastyczn� konfiguracj� scenariuszy, pozyskiwanie wsp�rz�dnych GPS, pr�dko�ci i przyspiesze� pojazd�w, a tak�e informacji o kolizjach i wykroczeniach drogowych. �rodowisko pozwala r�wnie� na modyfikacj� warunk�w pogodowych i pory dnia, co u�atwia testowanie algorytm�w w zr�nicowanych warunkach ruchu~\cite{carla}.

%=================================================================================================
\section{Architektura systemu CARLA}

Symulator CARLA bazuje na silniku graficznym Unreal Engine 4 (UE4), kt�ry odpowiada za realistyczne odwzorowanie �rodowiska symulacyjnego, w tym geometrii �wiata, pojazd�w, pieszych oraz warunk�w atmosferycznych~\cite{unreal-engine}. Na sil\textsf{}niku UE4 zbudowana jest skalowalna architektura typu klient�serwer, stanowi�ca podstaw� dzia�ania symulatora. 

Serwer odpowiada za wszystkie elementy zwi�zane z przebiegiem symulacji, w szczeg�lno�ci za renderowanie �wiata i aktor�w, obliczanie zjawisk fizycznych oraz generowanie pomiar�w z czujnik�w. Z kolei strona klienta sk�ada si� z modu��w kontroluj�cych logik� aktor�w na scenie i konfiguruj�cych warunki panuj�ce w �wiecie. Komunikacja mi�dzy klientem a serwerem odbywa si� za pomoc� interfejsu API CARLA (dost�pnego m.in. w Pythonie i C++), kt�ry jest stale rozwijany, aby udost�pnia� nowe funkcje~\cite{carla}.

Poni�ej przedstawiono wybrane elementy systemu:

\begin{itemize}
	\item Mened�er ruchu, czyli wbudowany system, kt�ry przejmuje kontrole nad pojazdami. Dzia�a jako przewodnik dostarczony przez CARLA do odtworzenia �rodowisk miejskich z realistycznymi zachowaniami.
	\item Czujniki, na kt�rych polegaj� pojazdy podczas przekazywania informacji o swoim otoczeniu. S� to specyficzni aktorzy pod��czeni do pojazdu a dane, kt�re otrzymuj� mog� by� przechowywane i wyszukiwane w celu u�atwienia procesu. Obecnie projekt obs�uguje ich r�ne typy - kamery, radary, lidary, itd.
	\item Rejestrator, czyli funkcja s�u��ca do odtwarzania symulacji krok po kroku dla ka�dego aktora na �wiecie. Dzi�ki niej u�ytkownik ma dost�p do ka�dego miejsca na �wiecie w dowolnym momencie osi czasu. 
	\item Most ROS i implementacja Autoware, kt�re zapewni� integracj� symulatora z innymi �rodowiskami do uczenia maszynowego i testowania jazdy autonomicznej.
	\item Otwarte zasoby, czyli u�atwienie tworzenia r�nych map miejskich z kontrol� warunk�w pogodowych i bibliotek� plan�w miast z szerokim zestawem aktor�w do wykorzystania. 
	\item Scenariusz jazdy. Aby u�atwi� proces uczenia i testowania pojazd�w autonomicznych, CARLA zapewnia seri� tras opisuj�cych r�ne sytuacje, kt�re mog� by� wielokrotnie wykorzystywane w kolejnych iteracjach eksperyment�w. Scenariusze te stanowi� r�wnie� podstaw� wyzwa� CARLA (CARLA Challenge), otwartych dla wszystkich u�ytkownik�w zainteresowanych ewaluacj� w�asnych rozwi�za� oraz por�wnaniem ich wynik�w w publicznych rankingach.
\end{itemize}

\section{Mo�liwo�ci symulatora}
CARLA wykorzystywana jest do badania trzech podej�� do autonomicznej jazdy:

I. Podej�cie klasyczne (modu�owe) � obejmuj�ce potok przetwarzania sk�adaj�cy si� z modu�u percepcji opartego na danych wizyjnych, planera trajektorii bazuj�cego na regu�ach oraz modu�u sterowania realizuj�cego manewry pojazdu.

II. Podej�cie oparte na uczeniu na�ladowczym (imitation learning) � wykorzystuj�ce g��bok� sie� neuronow�, kt�ra na podstawie danych sensorycznych generuje polecenia steruj�ce, ucz�c si� zachowa� kierowc�w na podstawie zarejestrowanych przyk�ad�w.

III. Podej�cie end-to-end z uczeniem ze wzmocnieniem � wykorzystuj�ce rozbudowan� sie� neuronow� trenowan� od pocz�tku do ko�ca w �rodowisku symulacyjnym z zastosowaniem metod uczenia ze wzmocnieniem.

Symulator Carla oferuje u�ytkownikowi mo�liwo�ci konfiguracji �rodowiska, w kt�rym odbywa si� symulacja. Mi�dzy innymi utworzenie w�asnej mapy z elementami w postaci budynk�w, pojazd�w oraz pieszych, czy te� korzystanie z gotowych �rodowisk utworzonych przez autor�w projektu. �rodowisko umo�liwia symulowanie warunk�w pogodowych, oraz sterowanie sygnalizacj� �wietln� za pomoc� funkcji opisanych w j�zyku Python. W �rodowisku Carla, zaimplementowane s� tak�e sensory, kt�re odgrywaj� istotn� rol� w przypadku kolizji czy te� ustawiania atrybut�w kamery. 

\subsection{�wiat}


Mapa jest jednym z g��wnych element�w �wiata symulatora. Zawiera zar�wno model 3D miejscowo�ci, jak i definicj� dr�g. Definicja dr�g na mapie oparta jest na pliku OpenDRIVE, standardowym formacie definicji dr�g z adnotacjami. Spos�b, w jaki standard 1.4 OpenDRIVE definiuje drogi, pasy ruchu, skrzy�owania itp. wp�ywa na funkcjonalno�� interfejsu API w j�zyku Python.

API Python dzia�a jako wysokopoziomowy system zapyta� do nawigacji po tych drogach. Jest ono stale rozwijane, aby zapewni� szerszy zestaw narz�dzi.

\begin{figure}[H]
	\centering
	\includegraphics[width=12cm]{Rysunki/Rozdzial2/map_1.png}
	\caption[Domy�lna mapa -  Town01]{Obraz mapy domy�lnej - Town01.}
	\label{fig:mapaTown01}
\end{figure} 

Domy�ln� map� symulatora CARLA jest Town01 Rys. \ref{fig:mapaTown01}, b�d�ca odwzorowaniem centrum du�ego miasta. Program oferuje wiele gotowych map, z kt�rych mo�na korzysta� w celu przeprowadzenia eksperyment�w, nie tylko podczas r�nych warunk�w pogodowych, ale r�wnie� w rozmaitych �rodowiskach. Przyk�adem jest mapa Town04 Rys. \ref{fig:mapaTown04}, przedstawiaj�ca ma�e miasteczko w g�rach, przy jeziorze.

\begin{figure}[H]
	\centering
	\includegraphics[width=12cm]{Rysunki/Rozdzial2/Town04.png}
	\caption[Mapa Town04]{Obraz mapy - Town04.}
	\label{fig:mapaTown04}
\end{figure} 

�wiat symulatora zapewnia r�wnie� aktor�w. Aktorami s� nie tylko pojazdy i piesi, ale tak�e czujniki, znaki drogowe, sygnalizacja �wietlna i widzowie. Bardzo wa�ne jest, aby mie� pe�ne zrozumienie, jak na nich operowa�. Istnieje mo�liwo�� tworzenia aktor�w zar�wno r�czenia, poprzez funkcj� \verb|spawn_actor()| jak i z wykorzystaniem przyk�adowego programu zapewnionego przez developer�w - \verb|generate_traffic.py|

\begin{figure}[H]
	\centering
	\includegraphics[width=12cm]{Rysunki/Rozdzial2/traffic.png}
	\label{fig:korek}
	\caption[Korek drogowy]{Obraz przedstawiaj�cy symulacj� 30 pojazd�w oraz 10 pieszych.}
\end{figure} 


\subsection{Sensory}


Sensor kolizji jest to czujnik, kt�ry rejestruje zdarzenie za ka�dym razem, gdy jego aktor macierzysty zderzy si� z czym� w �wiecie. Podczas jednego kroku symulacji mo�e zosta� wykrytych kilka kolizji. Aby zapewni� wykrywanie kolizji z dowolnym obiektem, serwer tworzy ,,fa�szywych'' aktor�w dla element�w takich jak budynki czy krzewy, dzi�ki czemu mo�liwe jest pobranie znacznika semantycznego w celu ich identyfikacji.

Detektory kolizji nie posiadaj� �adnych konfigurowalnych atrybut�w. \\


\begin{table}[H]
	\centering
	\begin{tabular}{|l|l|l|} 
		\hline
		Atrybuty        & Typ             & Opis                                                                                                            \\ 
		\hline
		\verb|frame|           & int             & Numer ramki, w kt�rej dokonano pomiaru.                                                                         \\ 
		\hline
		\verb|timestamp|       & double          & \begin{tabular}[c]{@{}l@{}}Czas symulacji pomiaru w\\sekundach od jej pocz�tku.\end{tabular}                    \\ 
		\hline
		\verb|transform|       & carla.Transform & \begin{tabular}[c]{@{}l@{}}Po�o�enie i obr�t we wsp�rz�dnych \\�wiata czujnika w czasie pomiaru.\end{tabular}  \\ 
		\hline
		\verb|actor|           & carla.Actor     & \begin{tabular}[c]{@{}l@{}}Aktor, kt�ry zmierzy� kolizj� \\(rodzic czujnika).\end{tabular}                      \\ 
		\hline
		\verb|other_actor|    & carla.Actor     & Aktor, z kt�rym zderzy� si� rodzic.                                                                             \\ 
		\hline
		\verb|normal_impulse| & carla.Vector3D  & Normalny impuls wynikaj�cy z kolizji.                                                                           \\
		\hline
	\end{tabular}
	\caption{Atrybuty sensora kolizji}
\end{table}

\begin{figure}[h!]
	\centering
	\includegraphics[width=12cm]{Rysunki/Rozdzial2/collision_example.png}
	\label{fig:przyklad3D}
	 \caption[Sensor odpowiadaj�cy za kolizje]{Przyk�adowy obraz kolizji ze s�upem.}
\end{figure}

Kolejnym sensorem jest kamera g��bi. Kamera dostarcza surowe dane sceny koduj�ce odleg�o�� ka�dego piksela od kamery (znane r�wnie� jako bufor g��bi lub z-bufor) w celu stworzenia mapy g��bi element�w.

Obraz koduje warto�� g��bi na piksel u�ywaj�c 3 kana��w przestrzeni kolor�w RGB, od mniej do bardziej znacz�cych bajt�w: R -> G -> B. Rzeczywista odleg�o�� w metrach mo�e by� zdekodowana za pomoc�: \newline

\begin{lstlisting}[style=praca]
	normalized = (R + G * 256 + B * 256 * 256) / (256 * 256 * 256 - 1)
	in_meters = 1000 * normalized
\end{lstlisting}


\begin{figure}[H]
	\centering
	\includegraphics[width=12cm]{Rysunki/Rozdzial2/depth_camera_oryginal.png}
	\label{fig:przyklad3D}
	\caption[Kamera g��bi orygina�]{Oryginalny obraz pochodz�cy z kamery g��bi.}
\end{figure}

Wyj�ciowy obraz CARLA.Image powinien zosta� zapisany na dysk przy u�yciu \verb|carla.colorConverter|, kt�ry zamieni odleg�o�� zapisan� w kana�ach RGB na [0,1] float zawieraj�cy odleg�o��, a nast�pnie przet�umaczy to na skal� szaro�ci. Istniej� dwie opcje w \verb|carla.colorConverter|, aby uzyska� widok g��bi: g��bia w odcieniach szaro�ci oraz g��boko�� logarytmiczna. Precyzja jest milimetrowa w obu, ale podej�cie logarytmiczne zapewnia lepsze wyniki dla bli�szych obiekt�w. Ponadto widoczno�� jest lepsza dla u�ytkownika.

\begin{figure}[H]
	\centering
	\includegraphics[width=12cm]{Rysunki/Rozdzial2/depth_gray_scale.png}
	\label{fig:przyklad3D}
	\caption[Kamera g��bi po konwersji]{Obraz pochodz�cy z kamery g��bi po konwersji w odcieniach szaro�ci.}
\end{figure} 


\begin{figure}[H]
	\centering
	\includegraphics[width=12cm]{Rysunki/Rozdzial2/depth_logarytmic.png}
	\label{fig:przyklad3D}
	\caption[Kamera g��bi logarytmiczna]{Obraz pochodz�cy z kamery g��bi - logarytmiczny.}
\end{figure} 


Sensor GNSS (Global Navigation Satellite Systems) - podaje aktualn� pozycj� GNSS swojego obiektu nadrz�dnego. Jest ona obliczana poprzez dodanie pozycji metrycznej do pocz�tkowej lokalizacji georeferencyjnej zdefiniowanej w definicji mapy OpenDRIVE

\begin{figure}[H]
	\centering
	\includegraphics[width=12cm]{Rysunki/Rozdzial2/gnss.png}
	\label{fig:gnss}
	\caption[Sensor GNSS]{Widok sensora GNSS.}
\end{figure} 

Sensor IMU dostarcza natomiast miary, kt�re akcelerometr, �yroskop i kompas mog�yby pobra� dla obiektu nadrz�dnego. Dane s� pobierane z bie��cego stanu obiektu.

\begin{figure}[H]
	\centering
	\includegraphics[width=12cm]{Rysunki/Rozdzial2/imu.png}
	\label{fig:imu}
	\caption[Sensor IMU]{Widok sensora IMU.}
\end{figure} 

Detektor wtargni�cia na pas ruchu rejestruje zdarzenie za ka�dym razem, gdy jego rodzic przekroczy oznaczenie pasa ruchu. Czujnik wykorzystuje dane drogowe dostarczane przez opis mapy OpenDRIVE, aby okre�li�, czy pojazd macierzysty naje�d�a na inny pas ruchu, bior�c pod uwag� przestrze� mi�dzy ko�ami. Nale�y jednak zwr�ci� uwag� na nast�puj�ce kwestie:

Rozbie�no�ci pomi�dzy plikiem OpenDRIVE a map� spowoduj� powstanie nieprawid�owo�ci, takich jak przecinaj�ce si� pasy ruchu, kt�re nie s� widoczne na mapie.
Wyj�cie pobiera list� oznacze� przecinaj�cych si� pas�w: obliczenia s� wykonywane w OpenDRIVE i uwzgl�dniaj� ca�� przestrze� pomi�dzy czterema ko�ami jako ca�o��. W zwi�zku z tym, w tym samym czasie mo�e by� przekraczanych wi�cej ni� jeden pas ruchu.

\begin{figure}[H]
	\centering
	\includegraphics[width=12cm]{Rysunki/Rozdzial2/crossed_line.png}
	\label{fig:crossedline}
	\caption[Sensor pDrzeci�cia linii]{Obraz przyk�adu przeci�cia linii.}
\end{figure} 

Sensor LIDAR (eng. Light Detection and Ranging) -czujnik symuluj�cy obrotowy skaner laserowy, w kt�rym generowanie pomiar�w realizowane jest z wykorzystaniem metody ray castingu. Punkty generowane s� poprzez emisj� promieni laserowych dla ka�dego kana�u, rozmieszczonych w zakresie pionowego pola widzenia (FOV). Obr�t jest symulowany poprzez obliczenie k�ta poziomego, o jaki obr�ci� si� LIDAR w danej klatce. Chmura punkt�w generowana jest poprzez wykonanie operacji ray castingu dla ka�dego promienia lasera w ka�dym kroku symulacji.

Pomiar LIDAR-owy zawiera paczk� z wszystkimi punktami wygenerowanymi w przedziale czasu 1/FPS. Podczas tego interwa�u fizyka nie jest aktualizowana, wi�c wszystkie punkty w pomiarze odzwierciedlaj� ten sam "statyczny obraz" sceny.

Informacja z pomiaru LIDAR-owego jest zakodowana w postaci punkt�w 4D. Pierwsze trzy z nich to punkty przestrzenne we wsp�rz�dnych xyz, a ostatni to straty intensywno�ci podczas jazdy. Intensywno�� ta jest obliczana wed�ug nast�puj�cego wzoru:

\begin{equation}
	\frac{I}{I_0} = e^{-a \cdot d}
\end{equation}

Gdzie:

\begin{description}
	\item[$a$] -- Oznacza wsp�czynnik t�umienia. Mo�e on zale�e� od d�ugo�ci fali czujnika oraz warunk�w atmosferycznych. Mo�na go zmodyfikowa� za pomoc� atrybutu LIDAR \verb|atmosphere_attenuation_rate|.
	\item[$b$] -- Odleg�o�� od punktu trafienia do czujnika.
\end{description}

W celu zwi�kszenia realizmu symulacji LIDAR umo�liwia losowe odrzucanie punkt�w chmury (general drop-off), co pozwala modelowa� straty pomiarowe oraz poprawi� wydajno�� obliczeniow�. Ustawienie parametru na warto�� 0,5 powoduje odrzucenie 50\% punkt�w.
\begin{lstlisting}[style=praca]
	dropoff\_general\_rate = 0.5)
\end{lstlisting}

Drugim mechanizmem jest zrzucanie punkt�w zale�ne od intensywno�ci odbicia. Dla ka�dego wykrytego punktu wykonywana jest dodatkowa operacja odrzucenia z prawdopodobie�stwem wyznaczanym na podstawie obliczonej intensywno�ci sygna�u. Prawdopodobie�stwo to definiowane jest przez dwa parametry: \verb|dropoff_zero_intensity|, okre�laj�cy prawdopodobie�stwo odrzucenia punkt�w o zerowej intensywno�ci, oraz \verb|dropoff_intensity_limit|,  wyznaczaj�cy pr�g intensywno�ci, powy�ej kt�rego punkty nie s� odrzucane. W zakresie pomi�dzy tymi warto�ciami prawdopodobie�stwo zrzucenia punktu zmienia si� liniowo jako funkcja intensywno�ci. \\

Dodatkowo, atrybut \verb|noise_stddev| tworzy model szumu, aby symulowa� nieoczekiwane odchylenia, kt�re pojawiaj� si� w rzeczywistych czujnikach. Dla warto�ci dodatnich, ka�dy punkt jest losowo zak��cany wzd�u� wektora promienia lasera. W rezultacie otrzymujemy czujnik LIDAR z doskona�ym pozycjonowaniem k�towym, ale zaszumionym pomiarem odleg�o�ci.

\begin{figure}[H]
	\centering
	\includegraphics[width=\textwidth]{Rysunki/Rozdzial2/lidar.png}
	\label{fig:SensorLIDAR}
	\caption[Sensor LIDAR]{Obraz sensora LIDAR.}
\end{figure} 

Detektor przeszk�d rejestruje zdarzenie za ka�dym razem, gdy aktor macierzysty ma przed sob� przeszkod�. W celu przewidywania przeszk�d, czujnik tworzy przed pojazdem macierzystym kszta�t kapsu�y i wykorzystuje go do sprawdzania, czy nie dochodzi do kolizji. Aby zapewni� wykrywanie kolizji z ka�dym rodzajem obiektu, serwer tworzy "fa�szywych" aktor�w dla element�w takich jak budynki lub krzewy, dzi�ki czemu mo�na pobra� znacznik semantyczny w celu ich identyfikacji.

\begin{figure}[H]
	\centering
	\includegraphics[width=6cm]{Rysunki/Rozdzial2/obstacle_detector.png}
	\label{fig:SensorDetekcji}
	\caption[Sensor detekcji obiekt�w]{Obraz sensora detekcji obiekt�w.}
\end{figure} 

Sensor radaru jest czujnikiem generuj�cym sto�kowy obszar detekcji, w obr�bie kt�rego wykrywane s� obiekty znajduj�ce si� w zasi�gu sensora. Dane radarowe reprezentowane s� w postaci dwuwymiarowej mapy punkt�w, zawieraj�cej informacje o po�o�eniu wykrytych element�w oraz ich pr�dko�ci wzgl�dnej wzgl�dem czujnika. Informacje te mog� by� wykorzystywane do analizy ruchu obiekt�w, okre�lania ich kierunku oraz oceny dynamiki sceny. Ze wzgl�du na zastosowanie wsp�rz�dnych biegunowych, g�sto�� punkt�w jest najwi�ksza w pobli�u osi widzenia sensora.

\begin{figure}[H]
	\centering
	\includegraphics[width=\textwidth]{Rysunki/Rozdzial2/radar_sensor.png}
	\label{fig:przyklad3D}
	\caption[Sensor radaru]{Obraz sensora radaru.}
\end{figure} 

% biblio
%https://carla.readthedocs.io/en/latest/ref_sensors/
%

\subsection{Pogoda}

Pogoda w symulatorze CARLA nie jest reprezentowana jako odr�bna klasa, lecz jako zbi�r parametr�w �rodowiskowych dost�pnych w �wiecie symulacji. Parametry te obejmuj� mi�dzy innymi po�o�enie s�o�ca, stopie� zachmurzenia, si�� wiatru, mg��, deszcz oraz �nieg, co umo�liwia testowanie algorytm�w w szerokim zakresie warunk�w atmosferycznych. Aby zdefiniowa� w�asn� pogod�, wykorzystuje si� klas� pomocnicz� \verb|carla.WeatherParameters|, w kt�rej mo�na ustawi� odpowiednie warto�ci atrybut�w odpowiadaj�cych za warunki atmosferyczne.

\begin{lstlisting}[style=praca, caption={Przyk�adowy kod ustawiania parametr�w pogody w symulatorze CARLA.}]
	weather = carla.WeatherParameters(
	cloudiness=80.0,
	precipitation=30.0,
	sun_altitude_angle=70.0
	)
	world.set_weather(weather)
	print(world.get_weather())
\end{lstlisting}


\begin{figure}[H]
	\centering
	\includegraphics[width=\textwidth]{Rysunki/Rozdzial2/soft_rain_sunset.png}
	\label{fig:lekki_deszcz_zach�d}
	\caption[Lekki deszcz o zachodzie s�o�ca]{Obraz przedstawiaj�cy lekki deszcz o zachodzie s�o�ca.}
\end{figure} 


Dost�pne s� r�wnie� gotowe ustawienia pogody, kt�re mo�na bezpo�rednio zastosowa� w symulacji jazdy samochodem. Lista zawieraj�ca przygotowane warunki drogowe znajduje si� w klasie \verb|carla.WeathrrParameters|.

\begin{lstlisting}[style=praca]
	world.set_weather(carla.WeatherParameters.WetCloudySunset)
\end{lstlisting}

\begin{figure}[H]
	\centering
	\includegraphics[width=12cm]{Rysunki/Rozdzial2/cloudy_night.png}
	\label{fig:zachmurzona_noc}
	\caption[Zachmurzona noc]{Obraz przedstawiaj�cy zachmurzon� noc.}
\end{figure} 

Istnieje r�wnie� mo�liwo�� dostosowania pogody za pomoc� dw�ch skrypt�w dostarczanych w pakiecie symulatora CARLA. 
S� to:
\begin{itemize}
	\item \verb|environment.py| (in PythonAPI/util) � Zapewnia dost�p do parametr�w pogodowych i o�wietleniowych, dzi�ki czemu mo�na je zmienia� w czasie rzeczywistym.
	\item \verb|dynamic_weather.py| (in PythonAPI/examples) � W��cza okre�lony cykl pogodowy przygotowany przez deweloper�w dla ka�dej mapy CARLA.
\end{itemize}

\begin{figure}[H]
	\centering
	\includegraphics[width=12cm]{Rysunki/Rozdzial2/hard_rain_noon.png}
	\label{fig:przyklad3D}
	\caption[Mocny deszcz w po�udnie]{Obraz przedstawiaj�cy mocny deszcz w po�udnie.}
\end{figure} 
%https://carla.readthedocs.io/en/latest/core_world/#weather


\subsection{O�wietlenie}

�wiat�a uliczne w��czaj� si� automatycznie, gdy symulacja przechodzi w tryb nocny. �wiat�a zosta�y umieszczone przez tw�rc�w mapy i s� dost�pne jako obiekty carla.Light. W�a�ciwo�ci takie jak kolor i nat�enie �wiat�a mog� by� dowolnie zmieniane. Zmienna \verb|light_state| typu \verb|carla.LightState| pozwala ustawi� je wszystkie w jednym wywo�aniu.
�wiat�a uliczne s� kategoryzowane za pomoc� ich atrybutu \verb|light_group|, typu \verb|carla.LightGroup|. Pozwala to na sklasyfikowanie �wiate� jako �wiat�a uliczne lub �wiat�a budynk�w. Aby obs�u�y� grupy �wiate� w jednym wywo�aniu, mo�na pobra� instancj� \verb|carla.LightManager|.

\begin{figure}[H]
	\centering
	\includegraphics[width=12cm]{Rysunki/Rozdzial2/street_lights.png}
	\label{fig:przyklad3D}
	\caption[�wiat�a uliczne]{Obraz przedstawiaj�cy �wiat�a uliczne.}
\end{figure} 

�wiat�a pojazdu musz� by� w��czane/wy��czane przez u�ytkownika. Ka�dy pojazd posiada zestaw �wiate� wymienionych w \verb|carla.VehicleLightState|. Jak dot�d, nie wszystkie pojazdy maj� zintegrowane �wiat�a. Poni�ej znajduje si� lista tych, kt�re s� dost�pne:
\begin{itemize}
	\item \verb|environment.py| (Rowery) � Wszystkie posiadaj� przednie i tylne �wiat�o pozycyjne.
	\item \verb|dynamic_weather.py| (Motocykle) � Modele Yamaha i Harley Davidson.
	\item \verb|dynamic_weather.py| (Samochody) - Audi TT, Chevrolet, Dodge (radiow�z), Audi e-tron, Lincoln, Mustang, Tesla 3S, Volkswagen T2 oraz nowi go�cie przybywaj�cy do CARLA.
\end{itemize}

�wiat�a pojazdu mog� by� pobierane i aktualizowane w dowolnym momencie za pomoc� metod \verb|carla.Vehicle.get_light_state| i \verb|carla.Vehicle.set_light_state|. U�ywaj� one operacji binarnych, aby dostosowa� ustawienie �wiate� ~\cite{carla}.

\begin{figure}[H]
	\centering
	\includegraphics[width=12cm]{Rysunki/Rozdzial2/vehicle_lights.png}
	\label{fig:przyklad3D}
	\caption[�wiat�a pojazdu]{Obraz przedstawiaj�cy �wiat�a pojazdu.}
\end{figure} 

\subsection{Reprezentacja obiekt�w w przestrzeni 3D i definicja Bounding Box}

Symulator CARLA \cite{carla} oraz Unreal Engine 4 (UE4) stosuj� lewoskr�tny uk�ad wsp�rz�dnych kartezja�ski z osi� Z skierowan� w g�r� (Z-up). Oznacza to:

\begin{itemize}
	\item \textbf{O� X}: skierowana do przodu (forward).
	\item \textbf{O� Y}: skierowana w prawo (right).
	\item \textbf{O� Z}: skierowana w g�r� (up).
\end{itemize}

Punkt (0, 0, 0) to pocz�tek uk�adu wsp�rz�dnych, zwany \textit{origin}. Warto�ci dodatnie na osi X oznaczaj� ruch do przodu, na osi Y w prawo, a na osi Z w g�r� \cite{devEpicgames}.

\subsection*{Jednostki miary}

W obu �rodowiskach:

\begin{itemize}
	\item \textbf{1 jednostka (unit)} odpowiada \textbf{1 centymetrowi} w rzeczywisto�ci.
	\item \textbf{1 metr} = \textbf{100 jednostek}.
	\item \textbf{1 kilometr} = \textbf{100\,000 jednostek}.
	\item \textbf{1 cal} = \textbf{2,54 jednostki}.
	\item \textbf{1 stopa} = \textbf{30,48 jednostki}.
\end{itemize}

Oznacza to, �e zar�wno w UE4, jak i w CARLA, 1 jednostka w grze to 1 cm w �wiecie rzeczywistym \cite{devEpicgamesUnits}.

\subsection*{Pozycjonowanie obiekt�w i punkt (0, 0, 0)}

W UE4 i CARLA:

\begin{itemize}
	\item \textbf{Punkt (0, 0, 0)}: znajduje si� w centrum �wiata gry, zwykle w miejscu, gdzie zaczyna si� scena.
	\item \textbf{Pozycjonowanie obiekt�w}: odbywa si� poprzez okre�lenie ich lokalizacji wzgl�dem tego punktu, np. \texttt{actor.set\_location(FVector(x, y, z))}.
\end{itemize}

	Ka�dy obiekt dynamiczny (tzw. aktor), taki jak pojazd czy pieszy, posiada swoj� pozycj� w globalnym uk�adzie wsp�rz�dnych �wiata, okre�lan� wzgl�dem punktu $(0,0,0)$ mapy. Kluczowym elementem w procesie detekcji i generowania danych ucz�cych jest zrozumienie, w jaki spos�b fizyczna bry�a obiektu jest reprezentowana numerycznie za pomoc� prostopad�o�cianu ograniczaj�cego, zwanego \textit{bounding box-em} \cite{carla, carla_bounding_boxes}.

\subsubsection{Relacja mi�dzy punktem Origin a �rodkiem geometrycznym}

	Ka�dy aktor w symulacji posiada sw�j punkt odniesienia, zwany \textit{Origin} lub \textit{Pivot Point}. W przypadku pojazd�w punkt ten jest zazwyczaj zlokalizowany na poziomie pod�o�a, w po�owie odleg�o�ci mi�dzy osiami k�, co u�atwia obliczenia fizyki jazdy. Nale�y jednak wyra�nie odr�ni� punkt \textit{Origin} aktora od �rodka jego obrysu (\textit{Bounding Box Center}).

Struktura \texttt{carla.BoundingBox} definiuje geometri� obiektu za pomoc� dw�ch kluczowych wektor�w:
\begin{enumerate}
	\item \texttt{location}: Okre�la przesuni�cie �rodka geometrycznego prostopad�o�cianu wzgl�dem punktu \textit{Origin} aktora. Wektor ten $(x_c, y_c, z_c)$ jest wyra�ony w lokalnym uk�adzie wsp�rz�dnych pojazdu. Przyk�adowo, dla samochodu osobowego �rodek bry�y znajduje si� zazwyczaj wy�ej (o� Z > 0) i mo�e by� przesuni�ty wzd�u� osi pod�u�nej wzgl�dem osi k�.
	\item \texttt{extent}: Jest to wektor $(e_x, e_y, e_z)$ okre�laj�cy po�ow� wymiar�w prostopad�o�cianu wzd�u� ka�dej z osi lokalnych. Oznacza to, �e ca�kowite wymiary obiektu wynosz� odpowiednio: d�ugo�� $2 \cdot e_x$, szeroko�� $2 \cdot e_y$ oraz wysoko�� $2 \cdot e_z$.
\end{enumerate}

Wsp�rz�dne wierzcho�k�w bounding boxa nie s� przechowywane wprost, lecz s� obliczane dynamicznie na podstawie jego �rodka oraz wektora \texttt{extent}. Dla lokalnego uk�adu odniesienia, wierzcho�ki $V_{local}$ zdefiniowane s� jako kombinacje dodawania i odejmowania warto�ci \texttt{extent} od �rodka \texttt{location} \cite{carla_bounding_boxes}.

%\begin{figure}[H]
%	\centering
%	% Tutaj wstaw grafik� ilustruj�c� wektory extent i relacj� origin-center
%	% Nazwij plik np. bbox_vectors_diagram.png
%	\includegraphics[width=0.8\textwidth]{Rysunki/Rozdzial2/bbox_vectors_diagram.png}
%	\caption{Schemat definicji Bounding Boxa: relacja mi�dzy punktem Origin aktora, �rodkiem Bounding Boxa (Center) oraz wektorem Extent okre�laj�cym wymiary bry�y.}
%	\label{fig:bbox_diagram}
%\end{figure}

\subsubsection{Transformacja do uk�adu wsp�rz�dnych �wiata}

Aby wyznaczy� po�o�enie wierzcho�k�w bounding boxa w globalnej przestrzeni 3D symulacji, konieczne jest wykonanie transformacji macierzowej. Proces ten uwzgl�dnia aktualn� pozycj� i rotacj� pojazdu na mapie. Wykorzystywana jest do tego macierz transformacji $M_{actor}$, kt�ra sk�ada si� z translacji (pozycji aktora wzgl�dem punktu 0,0,0 �wiata) oraz rotacji (k�t�w \textit{roll, pitch, yaw}).

Po�o�enie dowolnego wierzcho�ka $V_{world}$ w przestrzeni �wiata obliczane jest zgodnie z zale�no�ci�:
\begin{equation}
	V_{world} = M_{actor} \cdot V_{local}
\end{equation}
Gdzie $V_{local}$ to wsp�rz�dna wierzcho�ka w uk�adzie lokalnym pojazdu (uwzgl�dniaj�ca przesuni�cie \texttt{location} i wymiar \texttt{extent}). Operacja ta pozwala na precyzyjne umiejscowienie obrysu pojazdu w �wiecie 3D, niezale�nie od jego orientacji czy pochylenia wynikaj�cego z fizyki zawieszenia \cite{carla}.

\subsubsection{Rzutowanie z przestrzeni 3D na p�aszczyzn� obrazu 2D}

Ostatnim etapem, niezb�dnym do wygenerowania danych referencyjnych dla algorytmu YOLO (tzw. \textit{ground truth}), jest rzutowanie tr�jwymiarowych wierzcho�k�w bounding boxa na dwuwymiarow� p�aszczyzn� obrazu z kamery. Proces ten, realizowany m.in. w skrypcie \texttt{bounding\_boxes.py}, wymaga znajomo�ci parametr�w wewn�trznych i zewn�trznych kamery.

Rzutowanie odbywa si� w dw�ch krokach:
\begin{enumerate}
	\item \textbf{Transformacja �wiat $\rightarrow$ Kamera:} Punkty ze �wiata 3D s� przeliczane do uk�adu wsp�rz�dnych kamery przy u�yciu macierzy \textit{World-to-Camera} (b�d�cej odwrotno�ci� transformacji kamery w �wiecie).
	\item \textbf{Projekcja perspektywiczna:} Punkty z uk�adu kamery s� rzutowane na p�aszczyzn� obrazu przy u�yciu macierzy kalibracji $K$ (macierz parametr�w wewn�trznych), kt�ra zale�y od rozdzielczo�ci obrazu oraz k�ta widzenia (FOV).
\end{enumerate}

Wz�r opisuj�cy projekcj� punktu $P_{3D} = [x, y, z]^T$ na punkt obrazu $p_{2D} = [u, v]^T$ (we wsp�rz�dnych jednorodnych) przyjmuje posta�:
\begin{equation}
	s \begin{bmatrix} u \\ v \\ 1 \end{bmatrix} = K \cdot [R | t] \cdot \begin{bmatrix} x \\ y \\ z \\ 1 \end{bmatrix}
\end{equation}
Gdzie $K$ to macierz wewn�trzna kamery, $[R|t]$ to macierz transformacji z uk�adu �wiata do uk�adu kamery, a $s$ to wsp�czynnik skaluj�cy. Ostateczny prostok�t 2D (bbox 2D) wyznaczany jest poprzez znalezienie minimalnych i maksymalnych warto�ci wsp�rz�dnych $(u, v)$ spo�r�d wszystkich 8 rzutowanych wierzcho�k�w bry�y 3D. Takie podej�cie gwarantuje, �e wygenerowany obrys 2D w pe�ni obejmuje widoczny obiekt na zdj�ciu \cite{carla_bounding_boxes}.

\section{Instalacja symulatora CARLA}

\subsection{Wymagania sprz�towe}

Rekomendowana specyfikacja komputera dla najnowszej wersji symulatora CARLA  0.9.13 prezentuje si� nast�puj�co:
\begin{itemize}
	\item \verb|Procesor| Intel i7 gen 9th - 11th / Intel i9 gen 9th - 11th / AMD ryzen 7 / AMD ryzen 9
	\item \verb|Ilo�� pami�ci RAM| +16 GB pami�ci RAM
	\item \verb|Ilo�� miejsca na dysku| 130GB
	\item \verb|Karta graficzna| najlepiej z 6GB lub 8GB pami�ci  VRAM np. NVIDIA RTX 2070 / NVIDIA RTX 2080 / NVIDIA RTX 3070, NVIDIA RTX 3080 lub nowsze
	\item \verb|System operacyjny| Ubuntu 18.04/ 20.04/ 22.04/ Windows 10
\end{itemize}

\subsection{Instalacja i konfiguracja dla systemu Ubuntu}

Pierwszym etapem instalacji CARLA jest przygotowanie systemu. W pierwszej kolejno�ci system powinien zosta� zaktualizowany, a niezb�dne pakiety zainstalowane:

\begin{lstlisting}[caption={Aktualizacja systemu}]
	sudo apt update
	sudo apt upgrade
\end{lstlisting}

Nast�pnie musz� zosta� zainstalowane zale�no�ci wymagane przez CARLA:

\begin{lstlisting}[caption={Instalacja zale�no�ci}]
	sudo apt install build-essential clang cmake git libcurl4-openssl-dev libssl-dev \
	libsqlite3-dev libudev-dev pkg-config python3-dev python3-pip \
	python3-setuptools python3-wheel qt5-qmake qtbase5-dev libqt5core5a \
	libqt5gui5 libqt5widgets5 libprotobuf-dev protobuf-compiler \
	libsdl2-dev libpng-dev libjpeg-dev libtiff-dev libgtk-3-dev \
	libassimp-dev libblas-dev liblapack-dev libboost-all-dev \
	libopenblas-dev libatlas-base-dev libeigen3-dev
\end{lstlisting}

Dodatkowo, konieczne jest zainstalowanie j�zyka Python oraz mened�era pakiet�w PIP:

\begin{lstlisting}[caption={Instalacja Pythona i PIP}]
	sudo apt install python3 python3-pip
	python3 -m pip install --upgrade pip
\end{lstlisting}

\subsection{Instalacja Unreal Engine}

Ze wzgl�du na to, �e CARLA jest oparta na silniku Unreal Engine, konieczne jest jego zainstalowanie. Proces ten mo�e zosta� przeprowadzony za pomoc� Epic Games Launcher ~\cite{unreal-engine}.

\begin{enumerate}
	\item Nale�y za�o�y� konto na stronie \texttt{https://www.epicgames.com/}.
	\item Nast�pnie nale�y pobra� i zainstalowa� Epic Games Launcher.
	\item Po instalacji wymagane jest zalogowanie si� i przej�cie do zak�adki \texttt{Library}.
	\item Nale�y wybra� odpowiedni� wersj� Unreal Engine (zalecana 4.27 lub nowsza) i przeprowadzi� instalacj�.
	\item Po zako�czeniu instalacji Unreal Engine powinien zosta� uruchomiony za po�rednictwem Epic Games Launcher.
\end{enumerate}

\subsection{Pobranie �r�de� CARLA i kompilacja}

Aby pobra� CARLA, repozytorium musi zosta� sklonowane z GitHub:

\begin{lstlisting}[caption={Pobieranie �r�de� CARLA}]
	cd ~
	git clone https://github.com/carla-simulator/carla.git
	cd carla
\end{lstlisting}

Kolejnym krokiem jest przej�cie do katalogu zawieraj�cego projekt Unreal Engine dla CARLA:

\begin{lstlisting}[caption={Przechodzenie do folderu Unreal Engine dla CARLA}]
	cd Unreal/CarlaUE4
\end{lstlisting}

Nast�pnie konieczne jest uruchomienie Unreal Engine w celu konfiguracji projektu:

\begin{lstlisting}[caption={Uruchamianie Unreal Engine}]
	~/UnrealEngine/Engine/Binaries/Linux/UE4Editor CarlaUE4.uproject
\end{lstlisting}

Po uruchomieniu Unreal Engine nale�y wybra� opcj� \texttt{Generate Visual Studio project files} i zamkn�� edytor.

W kolejnym kroku wymagane jest powr�cenie do terminala i przeprowadzenie kompilacji:

\begin{lstlisting}[caption={Kompilacja CARLA}]
	make
\end{lstlisting}

\subsection{Uruchomienie symulatora CARLA}

Po zako�czeniu kompilacji symulator CARLA mo�e zosta� uruchomiony:

\begin{lstlisting}[caption={Uruchamianie CARLA}]
	cd ~/carla/Unreal/CarlaUE4/Binaries/Linux
	./CarlaUE4
\end{lstlisting}

Po pomy�lnym uruchomieniu symulatora powinien ukaza� si� obraz mapy domy�lnej widocznym na Rys. \ref{fig:mapaTown01}. Korzystanie z API CARLA w Pythonie Mo�na rozpocz�� poprzez instalacj� odpowiednich pakiet�w:

\begin{lstlisting}[caption={Instalacja zale�no�ci Pythona}]
	pip3 install carla
\end{lstlisting}

Aby zweryfikowa� poprawno�� dzia�ania, mo�na uruchomi� przyk�adowy skrypt Pythona:

\begin{lstlisting}[caption={Uruchamianie przyk�adowego skryptu}]
	cd ~/carla/PythonAPI/examples
	python spawn_npc.py
\end{lstlisting}

Je�eli wszystkie kroki zosta�y wykonane poprawnie, symulator CARLA powinien dzia�a�, a skrypt Python powinien uruchomi� si� bez problem�w.

\section{Skrypt do manualnego sterowania w symulatorze CARLA}
\label{sec:manual_control}

Skrypt \verb|manual_control.py| dostarczany wraz z symulatorem CARLA
umo�liwia r�czne sterowanie pojazdem z poziomu klawiatury oraz podgl�d
obrazu z kamer w czasie rzeczywistym. Stanowi on punkt wyj�cia do dalszej
integracji z zewn�trznymi algorytmami przetwarzania danych, takimi jak
systemy detekcji obiekt�w.

\subsection{Struktura skryptu}
Plik \verb|manual_control.py| zosta� zaprojektowany modularnie i obejmuje kilka kluczowych sekcji, z kt�rych ka�da realizuje specyficzny etap dzia�ania aplikacji:
\begin{itemize}
	\item \textbf{Importowanie bibliotek oraz konfiguracja klienta CARLA} � inicjalizacja po��czenia z serwerem symulacji oraz za�adowanie niezb�dnych bibliotek zewn�trznych, takich jak \verb|pygame| czy \verb|carla|.
	\item \textbf{Inicjalizacja �rodowiska i pojazdu} � stworzenie �wiata symulacyjnego, pojazdu oraz pod��czenie odpowiednich czujnik�w.
	\item \textbf{Obs�uga wej�cia u�ytkownika} � monitorowanie urz�dze� wej�ciowych (klawiatura, mysz) za pomoc� biblioteki \verb|pygame|.
	\item \textbf{G��wna p�tla gry (game loop)} � ci�g�e przetwarzanie zdarze�, aktualizacja stanu �wiata oraz renderowanie obrazu.
	\item \textbf{Zako�czenie i czyszczenie zasob�w} � prawid�owe usuni�cie wszystkich aktor�w i zwolnienie zasob�w systemowych.
\end{itemize}
Ka�dy z tych etap�w pe�ni fundamentaln� rol� w zapewnieniu poprawnej i wydajnej pracy symulacji.

\subsection{Logika dzia�ania programu \texttt{manual\_control.py}}
Program \verb|manual_control.py| realizuje pe�ny cykl �ycia aplikacji symulacyjnej: od inicjalizacji �rodowiska, przez obs�ug� sterowania r�cznego, a� po prawid�owe zako�czenie symulacji. Jego struktura zosta�a przedstawiona na schemacie blokowym (rysunek \ref{fig:manual_control_diagram}).

\begin{figure}[H]
	\centering
	\includegraphics[width=0.7\textwidth]{Rysunki/Rozdzial2/diagram_manual_control.png}
	\caption{Schemat blokowy dzia�ania pliku \texttt{manual\_control.py}}
	\label{fig:manual_control_diagram}
\end{figure}

G��wne komponenty systemu:
\begin{itemize}
	\item \textbf{Main Loop} � odpowiada za przetwarzanie zdarze� wej�ciowych, aktualizacj� stanu �wiata oraz renderowanie grafiki.
	\item \textbf{CameraManager} � umo�liwia zarz�dzanie widokami kamer i przetwarzanie danych z czujnik�w wizualnych.
	\item \textbf{World} � zarz�dza ca�ym �wiatem symulacji, w tym pojazdem u�ytkownika, pogod� oraz czujnikami.
	\item \textbf{KeyboardControl} � przetwarza sygna�y z klawiatury i przek�ada je na konkretne komendy steruj�ce pojazdem.
	\item \textbf{Sensors} � obejmuje wszystkie dost�pne czujniki w symulatorze (kamera RGB, czujniki kolizji, lane invasion sensor, GNSS, IMU, radar).
	\item \textbf{HUD (Heads-Up Display)} � odpowiedzialny za prezentowanie informacji o stanie symulacji w formie graficznej.
\end{itemize}

Komunikacja mi�dzy tymi komponentami odbywa si� w spos�b asynchroniczny, zapewniaj�c p�ynne dzia�anie symulacji oraz wysok� responsywno�� interfejsu u�ytkownika.
\subsection{G��wna p�tla gry - \texttt{game\_loop()}}
\label{subsec:game_loop}

G��wna logika dzia�ania programu \verb|manual_control.py| zosta�a
zaimplementowana w funkcji \verb|game_loop()|, kt�ra pe�ni rol� p�tli gry
(\textit{game loop}). W ka�dej iteracji p�tla ta odczytuje stan wej�cia
od u�ytkownika, aktualizuje sterowanie pojazdem oraz kontroluje tempo
symulacji, zapewniaj�c p�ynne dzia�anie programu.

\begin{lstlisting}[style=pythonColor, emph={Client, get\_world, VehicleControl, Clock, get\_pressed, parse\_control\_input, get\_actors, apply\_control, tick}, caption={Przyk�adowy kod p�tli gry -- \texttt{game\_loop()}}]
	def main():
	client = carla.Client('localhost', 2000)        # Inicjalizacja klienta CARLA
	world = client.get_world()                      # Pobranie �wiata symulacji
	control = carla.VehicleControl()                # Obiekt sterowania pojazdem
	
	pygame.init()                                   # Inicjalizacja pygame
	clock = pygame.time.Clock()                     # Zegar do kontrolowania FPS
	
	try:
	while True:
	keys = pygame.key.get_pressed()         # Pobieranie stanu klawiszy
	control = parse_control_input(keys)     # Analiza wej�cia od u�ytkownika
	
	vehicle = world.get_actors().filter('vehicle.*')[0]  # Pobranie pojazdu
	vehicle.apply_control(control)          # Zastosowanie sterowania
	
	clock.tick(30)                          # Ograniczenie liczby klatek na sekund�
	except KeyboardInterrupt:
	pass
\end{lstlisting}

Dzia�anie p�tli \verb|game_loop()| mo�na opisa� w nast�puj�cych krokach:
\begin{enumerate}
	\item \textbf{Inicjalizacja komponent�w}: tworzeni s� klient CARLA,
	obiekt �wiata \verb|world|, obiekt sterowania \verb|VehicleControl|
	oraz zegar \verb|Clock|, a biblioteka \verb|pygame| jest przygotowywana
	do obs�ugi wej�cia z klawiatury.
	
	\item \textbf{Pobranie stanu klawiszy}: funkcja
	\verb|pygame.key.get_pressed()| odczytuje aktualny stan wszystkich
	klawiszy na klawiaturze.
	
	\item \textbf{Przetwarzanie wej�cia u�ytkownika}: funkcja
	\verb|parse_control_input(keys)| interpretuje wci�ni�te klawisze (np.
	\verb|W|, \verb|S|, \verb|A|, \verb|D|, spacja) i na tej podstawie ustawia
	parametry \verb|throttle|, \verb|brake| oraz \verb|steer| w obiekcie
	\verb|carla.VehicleControl|.
	
	\item \textbf{Zastosowanie sterowania}: skonfigurowany obiekt
	\verb|VehicleControl| przekazywany jest do metody
	\verb|vehicle.apply_control(control)|, co powoduje wykonanie odpowiedniej
	akcji przez pojazd (przyspieszenie, hamowanie, skr�t).
	
	\item \textbf{Kontrola liczby klatek na sekund�}: funkcja
	\verb|clock.tick(30)| ogranicza cz�stotliwo�� wykonywania p�tli do
	30 iteracji na sekund�, co zapewnia p�ynno�� symulacji i stabilne
	warunki testowania.
\end{enumerate}

\subsection{Tryby asynchroniczny i synchroniczny w symulatorze CARLA}
\label{subsec:async_sync_modes}

Symulator CARLA umo�liwia uruchamianie �rodowiska w dw�ch podstawowych trybach: asynchronicznym oraz synchronicznym. W trybie asynchronicznym serwer symulacji samodzielnie aktualizuje �wiat z w�asn� cz�stotliwo�ci�, niezale�nie od dzia�a� klienta. W trybie synchronicznym ka�da klatka symulacji generowana jest dopiero po wywo�aniu odpowiedniej metody po stronie klienta, co pozwala �ci�le zsynchronizowa� stan �wiata z danymi z czujnik�w.

\subsubsection*{Tryb asynchroniczny (domy�lny)}

Tryb asynchroniczny jest ustawieniem domy�lnym po uruchomieniu symulatora CARLA i pod��czeniu klienta, takiego jak skrypt \verb|manual_control.py|. Serwer aktualizuje fizyk�, po�o�enie aktor�w oraz dane z czujnik�w z maksymaln� mo�liw� cz�stotliwo�ci�, niezale�nie od tego, jak szybko klient przetwarza otrzymywane informacje. P�tla gry po stronie klienta mo�e skupi� si� na obs�udze wej�cia u�ytkownika z klawiatury, aktualizacji interfejsu HUD oraz renderowaniu obrazu, podczas gdy serwer w tle nieprzerwanie prowadzi symulacj�. Wad� tego trybu jest brak gwarancji, �e obraz z kamer i inne pomiary b�d� dok�adnie odpowiada�y temu samemu krokowi czasowemu, co utrudnia precyzyjne gromadzenie danych do trenowania i ewaluacji algorytm�w detekcji obiekt�w.

W pocz�tkowej fazie przeprowadzanych eksperyment�w tryb asynchroniczny okaza� si� niewystarczaj�cy przy testowaniu algorytmu YOLO w czasie rzeczywistym. Serwer symulatora generowa� kolejne klatki szybciej, ni� klient by� w stanie je przetwarza�, co prowadzi�o do rozjechania si� czasowego danych: ramki detekcji odnosi�y si� do poprzednich stan�w �wiata, a nie do aktualnie renderowanego obrazu. W efekcie obrysy obiekt�w, takich jak piesi czy pojazdy, by�y rysowane z wyra�nym op�nieniem, cz�sto w miejscach, w kt�rych obiekt znajdowa� si� jedynie w poprzednich klatkach. Problem ten zosta� wyeliminowany dopiero po prze��czeniu symulacji w tryb synchroniczny, w kt�rym ka�da klatka obrazu jest jednoznacznie powi�zana z odpowiadaj�cymi jej wynikami detekcji.

%\begin{figure}[H]
%	\centering
%	% \includegraphics[width=\textwidth]{RysunkiRozdzial2_async_mode.png}
%	\caption{Widok symulatora CARLA w domy�lnym trybie asynchronicznym.}
%	\label{fig:carla_async_mode}
%\end{figure}

\subsubsection*{Tryb synchroniczny i uruchamianie z poziomu terminala}

Tryb synchroniczny zapewnia deterministyczne i powtarzalne dzia�anie symulacji, w kt�rym ka�da klatka jest jednoznacznie powi�zana z kompletem danych z czujnik�w. Klient wywo�uje metod� \verb|world.tick()| tylko wtedy, gdy jest gotowy na pobranie kolejnej pr�bki danych, co pozwala �ci�le zsynchronizowa� stan �wiata z obrazami z kamer, chmurami punkt�w LIDAR oraz innymi pomiarami. Jest to szczeg�lnie istotne przy przygotowywaniu zbior�w danych dla modeli detekcji obiekt�w, takich jak YOLO, gdzie dla ka�dej klatki obrazu wymagane s� dok�adnie odpowiadaj�ce jej adnotacje.

Prze��czenie symulacji w tryb synchroniczny wymaga jawnego wymuszenia tego ustawienia po stronie klienta. W kodzie odbywa si� to poprzez zmian� konfiguracji �wiata:

\begin{lstlisting}[style=pythonColor, emph={get\_settings, synchronous\_mode, fixed\_delta\_seconds, apply\_settings}, caption={Konfiguracja trybu synchronicznego w kodzie klienta}]
	settings = world.get_settings()
	settings.synchronous_mode = True
	settings.fixed_delta_seconds = 0.05  # np. 20 FPS
	world.apply_settings(settings)
\end{lstlisting}

Skrypt \verb|manual_control.py| udost�pnia w tym celu prze��cznik w linii polece�, kt�ry aktywuje tryb synchroniczny. Uruchomienie klienta z poziomu terminala mo�e wygl�da� nast�puj�co:

\begin{lstlisting}[caption={Uruchomienie skryptu manual\_control.py w trybie synchronicznym (Linux)}]
	python3 manual_control.py --sync
\end{lstlisting}

lub w systemie Windows:

\begin{lstlisting}[caption={Uruchomienie skryptu manual\_control.py w trybie synchronicznym (Windows)}]
	python manual_control.py --sync
\end{lstlisting}

Podanie parametru \verb|--sync| powoduje, �e w kodzie klienta ustawiany jest tryb synchroniczny oraz ewentualny sta�y krok czasowy, a nast�pnie wywo�ywana jest metoda \verb|world.apply_settings(settings)|, co prze��cza serwer w tryb synchroniczny. Zalet� takiej organizacji jest mo�liwo�� �cis�ego sprz�enia symulacji z zewn�trznymi algorytmami, kt�re mog� przetworzy� dane z bie��cej klatki i dopiero potem zainicjowa� przej�cie do nast�pnego kroku czasowego. Wad� jest natomiast to, �e ca�kowita pr�dko�� symulacji zale�y od wydajno�ci klienta � im d�u�ej trwa przetwarzanie danych, tym wolniej generowane s� kolejne klatki.

%\begin{figure}[H]
%	\centering
%	% \includegraphics[width=\textwidth]{RysunkiRozdzial2_sync_mode.png}
%	\caption{Widok symulatora CARLA w wymuszonym trybie synchronicznym.}
%	\label{fig:carla_sync_mode}
%\end{figure}

\subsection{Znaczenie dla integracji z algorytmami zewn�trznymi}

Zastosowanie trybu synchronicznego jest szczeg�lnie istotne w kontek�cie integracji z zewn�trznymi algorytmami analizy danych, np. systemami opartymi o YOLO. W takich przypadkach przetwarzanie obrazu musi by� zgodne ze stanem symulacji, aby wykrywane obiekty odpowiada�y ich rzeczywistej pozycji i pr�dko�ci. Brak synchronizacji mo�e prowadzi� do b��dnych detekcji i zaburzenia procesu wnioskowania.

\subsection{Sterowanie pojazdem z poziomu klawiatury}

Jednym z g��wnych element�w interakcji u�ytkownika z pojazdem w symulatorze CARLA jest sterowanie przy pomocy klawiatury. W tym celu skrypt \texttt{manual\_control.py} wykorzystuje bibliotek� \texttt{pygame}, kt�ra pozwala na obs�ug� wej�cia u�ytkownika i monitorowanie stanu klawiszy. G��wna funkcja odpowiedzialna za to zadanie to \texttt{parse\_control\_input()}, kt�ra analizuje naci�ni�te klawisze i na ich podstawie dostosowuje parametry sterowania pojazdem.

\begin{lstlisting}[style=pythonColor, emph={parse\_control\_input, get\_pressed, VehicleControl}, caption={Przyk�adowy kod obs�ugi sterowania pojazdem z klawiatury}]
	import pygame
	
	def parse_control_input():
	keys = pygame.key.get_pressed()       # Pobranie stanu klawiszy
	control = carla.VehicleControl()      # Obiekt sterowania pojazdem
	
	if keys[pygame.K_w]:                  # Przyspieszanie (gaz)
	control.throttle = 1.0
	if keys[pygame.K_s]:                  # Hamowanie
	control.brake = 1.0
	if keys[pygame.K_a]:                  # Skr�t w lewo
	control.steer = -1.0
	if keys[pygame.K_d]:                  # Skr�t w prawo
	control.steer = 1.0
	
	return control                        # Zwr�cenie obiektu sterowania
\end{lstlisting}

\noindent
Dzia�anie funkcji mo�na podsumowa� nast�puj�co:
\begin{itemize}
	\item \textbf{Pobieranie stanu klawiszy} � funkcja \verb|pygame.key.get_pressed()| zwraca tablic� warto�ci logicznych dla wszystkich klawiszy, umo�liwiaj�c detekcj� aktualnie wci�ni�tych przycisk�w.
	\item \textbf{Tworzenie obiektu sterowania} � obiekt \verb|carla.VehicleControl| przechowuje parametry steruj�ce pojazdem, takie jak \verb|throttle| (przyspieszenie), \verb|brake| (hamowanie) oraz \verb|steer| (k�t skr�tu).
	\item \textbf{Ustawianie parametr�w} � na podstawie wci�ni�tych klawiszy przypisywane s� odpowiednie warto�ci do w�a�ciwo�ci \verb|throttle|, \verb|brake| i \verb|steer|, np. klawisz \verb|W| powoduje przyspieszanie pojazdu, a \verb|S| uruchamia hamowanie.
	\item \textbf{Zwr�cenie obiektu sterowania} � po przetworzeniu stanu klawiatury funkcja zwraca skonfigurowany obiekt \verb|control|, kt�ry nast�pnie jest przekazywany do metody \verb|vehicle.apply_control(control)| odpowiedzialnej za wykonanie manewru w �wiecie symulacyjnym.
\end{itemize}

\subsection{Czyszczenie zasob�w (Cleanup)}
Po zako�czeniu dzia�ania symulacji niezb�dne jest prawid�owe usuni�cie wszystkich aktor�w, aby zwolni� pami�� oraz unikn�� potencjalnych b��d�w.

Proces ten obejmuje:
\begin{itemize}
	\item \textbf{Pobranie aktor�w} � funkcja \verb|world.get_actors()| zwraca wszystkie aktywne obiekty w �wiecie.
	\item \textbf{Zniszczenie aktor�w} � poprzez iteracyjne wywo�ywanie metody \verb|destroy()| na ka�dym z aktor�w.
\end{itemize}

Przyk�adowy kod funkcji czyszcz�cej:

\begin{lstlisting}[style=pythonColor, emph={cleanup, get\_actors, destroy}, caption={Kod funkcji odpowiedzialnej za czyszczenie zasob�w}]
	def cleanup(world):
	actors = world.get_actors()
	for actor in actors:
	actor.destroy()
\end{lstlisting}

Poprawne czyszczenie zasob�w zapewnia optymaln� wydajno�� symulacji i umo�liwia jej wielokrotne uruchamianie bez ryzyka narastania b��d�w pami�ciowych.
\chapter{System rozpoznawania obraz�w}

\textbf{Rozpoznawanie obraz�w} to dziedzina komputerowego przetwarzania danych, kt�rej celem jest identyfikacja i klasyfikacja obiekt�w w obrazach cyfrowych. W ostatnich latach, dzi�ki rozwojowi g��bokiego uczenia (deep learning) oraz sieci neuronowych, osi�gni�to znacz�cy post�p w tej dziedzinie.

\textbf{Sieci neuronowe} to modele matematyczne inspirowane struktur� ludzkiego m�zgu, sk�adaj�ce si� z po��czonych ze sob� neuron�w (w�z��w), kt�re przetwarzaj� informacje. G��bokie uczenie odnosi si� do sieci neuronowych o wielu warstwach (tzw. g��bokich sieci), kt�re potrafi� uczy� si� reprezentacji danych na r�nych poziomach abstrakcji. W kontek�cie rozpoznawania obraz�w, najcz�ciej stosuje si� konwolucyjne sieci neuronowe (Convolutional Neural Networks, CNN), kt�re s� szczeg�lnie efektywne w analizie danych obrazowych.

Poni�ej przedstawiono \textbf{przegl�d kluczowych architektur} stosowanych w rozpoznawaniu obraz�w, wraz z ich charakterystyk� i przyk�adami.

\begin{figure}[htbp]
	\centering
	\vspace{-0.75\baselineskip}
	\includegraphics[width=12cm]{Rysunki/Rozdzial3/RCNN.png}
	\caption[RCNN]{Schemat dzia�ania R-CNN.}
	\label{fig:RCNN}
\end{figure}

\begin{itemize}
	\item \textbf{R-CNN (Region-based Convolutional Neural Networks)} to jedna z pierwszych skutecznych metod detekcji obiekt�w, oparta na analizie wybranych region�w obrazu. W pierwszym etapie stosowany jest algorytm \textit{Selective Search}, kt�ry generuje oko�o 2000 propozycji obszar�w potencjalnie zawieraj�cych obiekty, uwzgl�dniaj�c cechy takie jak kolor, tekstura czy kszta�t. Ka�dy region jest nast�pnie skalowany i przetwarzany przez sie� konwolucyjn� (CNN), co pozwala na ekstrakcj� cech wizualnych. Ostatecznie, klasyfikator SVM decyduje o przynale�no�ci danego regionu do konkretnej klasy, a regresor liniowy doprecyzowuje wsp�rz�dne ramki ograniczaj�cej.~\cite{ren2015faster}
	
	Cho� metoda ta cechuje si� wysok� precyzj� detekcji, jej g��wn� wad� pozostaje du�e zapotrzebowanie obliczeniowe zwi�zane z konieczno�ci� analizowania tysi�cy region�w dla ka�dego obrazu.
	
	\item \textbf{Fast R-CNN} to ulepszona wersja modelu R-CNN, zaprojektowana z my�l� o zwi�kszeniu wydajno�ci i redukcji czasoch�onno�ci procesu detekcji obiekt�w. G��wne usprawnienie polega na zastosowaniu jednej sieci konwolucyjnej (CNN) do ekstrakcji cech z ca�ego obrazu przed wygenerowaniem propozycji region�w, co eliminuje konieczno�� wielokrotnego przetwarzania tych samych obszar�w.
	
	Zamiast klasyfikatora SVM, Fast R-CNN wykorzystuje wbudowan� warstw� softmax do rozpoznawania klas obiekt�w oraz regresj� do dok�adnej lokalizacji ramek ograniczaj�cych. Regiony zainteresowania (RoI) s� przekszta�cane za pomoc� operacji \textit{RoI Pooling}, kt�ra dostosowuje ich wymiary do jednolitego formatu wej�ciowego dla dalszej klasyfikacji.
	
	Dzi�ki integracji tych rozwi�za� Fast R-CNN znacznie skraca czas przetwarzania i umo�liwia efektywne wykrywanie obiekt�w przy zachowaniu wysokiej dok�adno�ci.~\cite{girshick2015fast}
	
	\begin{figure}[H]
		\centering
		\includegraphics[width=12cm]{Rysunki/Rozdzial3/fastRCNN.png}
		\caption[fastRCNN]{Schemat dzia�ania Fast R-CNN. �r�d�o: \cite{stutzFastRCNN}.}
		\label{fig:fastRcnn}
	\end{figure}
	
	\item \textbf{Faster R-CNN} to zaawansowana architektura detekcji obiekt�w, kt�ra integruje Sie� Generuj�c� Propozycje Region�w (Region Proposal Network, RPN) z g��wn� sieci� konwolucyjn�. Kluczow� innowacj� jest eliminacja potrzeby korzystania z zewn�trznych algorytm�w generuj�cych propozycje region�w, co znacznie skraca czas detekcji i zwi�ksza efektywno��~\cite{ren2015faster}.
	
	Ca�y obraz jest najpierw przetwarzany przez CNN w celu uzyskania map cech. Nast�pnie RPN generuje regiony zainteresowania bezpo�rednio na podstawie tych map, przewiduj�c zar�wno ich pozycje, jak i prawdopodobie�stwo zawierania obiekt�w. Regiony te s� klasyfikowane oraz doprecyzowywane przez ko�cowy modu� sieci. Dzi�ki wsp�dzieleniu warstw konwolucyjnych pomi�dzy RPN a modu�em klasyfikacyjnym, Faster R-CNN oferuje wysok� dok�adno�� przy znacznie lepszej wydajno�ci ni� jego poprzednicy.
	
	\begin{figure}[H]
		\centering
		\includegraphics[width=12cm]{Rysunki/Rozdzial3/fasterRCNN2.png}
		\caption[fasterRCNN2]{Schemat dzia�ania Faster R-CNN.}
		\label{fig:fasterRcnn2}
	\end{figure}

	\item \textbf{SSD (Single Shot MultiBox Detector)} to jednoetapowa metoda detekcji obiekt�w, kt�ra, w przeciwie�stwie do model�w opartych na regionach, takich jak R-CNN, wykonuje detekcj� i klasyfikacj� obiekt�w w jednym kroku. Dzi�ki temu SSD jest szybsza i bardziej efektywna, co czyni j� odpowiedni� do zastosowa� wymagaj�cych przetwarzania w czasie rzeczywistym.
	
	\begin{figure}[H]
		\centering
		\includegraphics[width=12cm]{Rysunki/Rozdzial3/ssd_architecture.png}
		\label{fig:ssd_architecture}
		\caption[SSD]{Schemat architektury SSD \cite{liu2016ssd}.}
	\end{figure} 
	
	SSD wykorzystuje konwolucyjne sieci neuronowe (CNN) do ekstrakcji cech obrazu i jednoczesnej predykcji po�o�enia oraz klasy obiekt�w na wielu skalach. Model sk�ada si� z:
	\begin{itemize}
		\item \textbf{Sieci bazowej} � najcz�ciej stosuje si� architektur� VGG16 bez w pe�ni po��czonych warstw,
		\item \textbf{Dodatkowych warstw konwolucyjnych} � umo�liwiaj� one detekcj� obiekt�w na r�nych poziomach szczeg�owo�ci,
		\item \textbf{Mechanizmu MultiBox} � generuje wiele ramki ograniczaj�cych (bounding boxes) o r�nych proporcjach i rozmiarach,
		\item \textbf{Funkcji straty} � sk�ada si� z dw�ch komponent�w: b��du klasyfikacji (cross-entropy loss) oraz b��du lokalizacji (smooth L1 loss).
	\end{itemize}
	
	Dzi�ki swojej szybko�ci i efektywno�ci SSD znajduje zastosowanie w wielu dziedzinach, takich jak:
	\begin{itemize}
		\item Systemy monitoringu wizyjnego,
		\item Rozpoznawanie obiekt�w w autonomicznych pojazdach,
		\item Aplikacje rzeczywisto�ci rozszerzonej (AR),
		\item Systemy wspomagania kierowcy (ADAS).
	\end{itemize}
	
	\item \textbf{YOLO (You Only Look Once)} to kolejna jednokrokowa architektura, kt�ra traktuje detekcj� obiekt�w jako problem regresji, przewiduj�c bezpo�rednio klasy i po�o�enie obiekt�w w obrazie. Dzi�ki temu YOLO osi�ga bardzo wysok� szybko�� detekcji, co jest istotne w aplikacjach wymagaj�cych przetwarzania w czasie rzeczywistym.
	
	\item \textbf{DETR (Detection Transformer)} to nowoczesna architektura detekcji obiekt�w oparta na transformerach, kt�ra integruje mechanizmy uwagi (attention mechanisms) w procesie detekcji. DETR eliminuje potrzeb� stosowania tradycyjnych metod generowania propozycji region�w, co upraszcza proces detekcji i pozwala na bardziej efektywne wykorzystanie danych.
\end{itemize}

Poni�sza tabela przedstawia \textbf{por�wnanie om�wionych architektur} pod wzgl�dem szybko�ci i dok�adno�ci detekcji:

\begin{table}[H]
	\centering
	\resizebox{\textwidth}{!}{%
		\begin{tabular}{|l|l|l|l|}
			\hline
			Architektura & Szybko�� (FPS)              & Dok�adno�� (mAP) & Zastosowanie                         \\ 
			\hline
			R-CNN        &  1 FPS                      & Wysoka           & Analiza offline                      \\ 
			\hline
			Fast R-CNN   & \textasciitilde{}2-3 FPS    & Wysoka           & Analiza offline                      \\ 
			\hline
			Faster R-CNN & \textasciitilde{}5-10 FPS   & Bardzo wysoka    & Zastosowania wymagaj�ce dok�adno�ci  \\ 
			\hline
			SSD          & \textasciitilde{}20-60 FPS  & �rednia          & Zastosowania w czasie rzeczywistym   \\ 
			\hline
			YOLO         & \textasciitilde{}45-150 FPS & Wysoka           & Wykrywanie w czasie rzeczywistym     \\ 
			\hline
			DETR         & \textasciitilde{}10-20 FPS  & Bardzo wysoka    & Nowoczesne zastosowania AI           \\
			\hline
		\end{tabular}
	}
	\caption{Por�wnanie wybranych architektur}
\end{table}

Rozpoznawanie obraz�w z wykorzystaniem sieci neuronowych jest obecnie jedn� z kluczowych technologii w dziedzinie sztucznej inteligencji. Dzi�ki zastosowaniu zaawansowanych architektur, takich jak R-CNN, SSD, YOLO czy DETR, mo�liwe jest osi�ganie wysokiej dok�adno�ci i szybko�ci detekcji obiekt�w. Wyb�r odpowiedniej architektury zale�y od specyficznych wymaga� aplikacji, takich jak potrzeba przetwarzania w czasie rzeczywistym, dost�pno�� zasob�w obliczeniowych oraz dok�adno�� rozpoznawania.

Przysz�y rozw�j w tej dziedzinie prawdopodobnie b�dzie koncentrowa� si� na dalszym zwi�kszaniu efektywno�ci modeli, integracji z systemami opartymi na transformerach oraz rozwijaniu metod rozpoznawania obiekt�w w trudnych warunkach �rodowiskowych.
\section{YOLOv4 i konwolucyjne sieci neuronowe}

YOLOv4 (\textit{You Only Look Once version 4}) to jeden z najnowszych modeli s�u��cych do detekcji obiekt�w w obrazach i wideo, nale��cy do rodziny algorytm�w YOLO. Jego celem jest osi�gni�cie wysokiej precyzji przy jednoczesnym zachowaniu du�ej szybko�ci dzia�ania. YOLOv4 opiera si� na konwolucyjnych sieciach neuronowych (CNN), kt�re umo�liwiaj� ekstrakcj� cech wizualnych z obraz�w oraz ich klasyfikacj� \cite{yolov4_paper}.

\subsection{Konwolucyjne sieci neuronowe (CNN)}

Konwolucyjne sieci neuronowe (CNN) s� rodzajem sztucznych sieci neuronowych szczeg�lnie skutecznym w analizie obraz�w. Dzia�aj� one w spos�b hierarchiczny, w kt�rym r�ne warstwy sieci ucz� si� wykrywa� r�ne cechy obrazu � od prostych kraw�dzi i tekstur w pocz�tkowych warstwach, po bardziej z�o�one struktury, takie jak twarze czy obiekty w g��bszych warstwach. 

\begin{figure}[H]
	\centering
	\includegraphics[width=12cm]{Rysunki/Rozdzial3/cnn_structure.png}
	\caption{Schemat struktury konwolucyjnej sieci neuronowej~\cite{esezam}.}
	\label{fig:cnn}
\end{figure}

Konwolucyjne sieci neuronowe sk�adaj� si� z trzech g��wnych typ�w warstw:
\begin{itemize}
	\item \textbf{Warstwa konwolucyjna (Convolutional Layer)} � wykonuje operacj� konwolucji na obrazie wej�ciowym, wykrywaj�c podstawowe cechy, takie jak kraw�dzie, rogi i tekstury.
	\item \textbf{Warstwa aktywacji (Activation Layer)} � cz�sto stosuje funkcj� aktywacyjn� ReLU (Rectified Linear Unit), kt�ra wprowadza nieliniowo�� do sieci.
	\item \textbf{Warstwa poolingowa (Pooling Layer)} � zmniejsza rozmiar danych, podsumowuj�c cechy w danym obszarze, co prowadzi do zmniejszenia liczby parametr�w i zwi�kszenia wydajno�ci obliczeniowej.
\end{itemize}

Korzenie konwolucyjnych sieci neuronowych si�gaj� lat 80. XX wieku, kiedy Fukushima zaproponowa� model Neocognitron do rozpoznawania wzorc�w odpornych na przesuni�cia.\cite{fukushima1980} P�niejsze prace LeCuna i wsp�autor�w ugruntowa�y praktyczne zastosowania CNN w zadaniach takich jak rozpoznawanie cyfr i dokument�w.\cite{lecun1998}

\section{Zasada dzia�ania algorytmu YOLO i YOLOv4}

Algorytmy z rodziny \textit{YOLO} (You Only Look Once) nale�� do najwydajniejszych rozwi�za� w dziedzinie detekcji obiekt�w w czasie rzeczywistym, traktuj�c to zadanie jako problem regresyjny, w kt�rym obraz wej�ciowy jest bezpo�rednio odwzorowywany na zbi�r ramek detekcyjnych z przypisanymi klasami oraz wsp�czynnikami pewno�ci~\cite{Redmon2016,Bochkovskiy2020,Goodfellow2016}. W odr�nieniu od klasycznych detektor�w dwuetapowych, takich jak Faster R-CNN, modele \textit{YOLO} realizuj� proces detekcji w jednym przebiegu przez sie� neuronow� (\textit{one-stage detector}), co pozwala na osi�gni�cie wysokiej szybko�ci dzia�ania przy zachowaniu konkurencyjnej dok�adno�ci~\cite{Redmon2016,Bochkovskiy2020}. 

W algorytmach YOLO obraz wej�ciowy jest wst�pnie przeskalowywany do ustalonej rozdzielczo�ci (najcz�ciej \(416 \times 416\) lub \(608 \times 608\) pikseli), a nast�pnie przetwarzany przez g��bok� sie� konwolucyjn�, kt�ra jednocze�nie odpowiada za ekstrakcj� cech, lokalizacj� obiekt�w oraz ich klasyfikacj�~\cite{Bochkovskiy2020,Geron2019}. W wyniku dzia�ania sieci otrzymuje si� zbi�r propozycji ramek ograniczaj�cych (\textit{bounding boxes}) wraz z przypisanymi im prawdopodobie�stwami przynale�no�ci do klas oraz warto�ciami ufno�ci (\textit{objectness score}), kt�re nast�pnie s� filtrowane w celu uzyskania ko�cowych detekcji. YOLOv4 stanowi rozwini�cie wcze�niejszych wersji, wprowadzaj�c szereg modyfikacji architektonicznych i technik treningowych, zapewniaj�cych lepszy kompromis pomi�dzy dok�adno�ci� a szybko�ci� dzia�ania~\cite{Bochkovskiy2020}. 

\subsection*{Etap 1: Podzia� obrazu na siatk� i definicja anchor boxes}

Podstaw� dzia�ania algorytm�w YOLO jest podzia� przeskalowanego obrazu na regularn� siatk� przestrzenn� o rozmiarze \(S \times S\), gdzie \(S\) zale�y od poziomu rozdzielczo�ci detekcji~\cite{Redmon2016}. W przypadku YOLOv4 detekcja realizowana jest r�wnocze�nie na trzech skalach: \(13 \times 13\), \(26 \times 26\) oraz \(52 \times 52\), co umo�liwia skuteczne wykrywanie odpowiednio du�ych, �rednich oraz ma�ych obiekt�w~\cite{Bochkovskiy2020}. Ka�da kom�rka siatki odpowiada za wykrywanie obiekt�w, kt�rych �rodek masy znajduje si� w jej obr�bie, co ogranicza liczb� mo�liwych lokalizacji i upraszcza proces optymalizacji~\cite{Geron2019}. 

Dla ka�dej kom�rki siatki definiowanych jest \(B\) tzw. \textit{anchor boxes}, czyli z g�ry ustalonych propozycji ramek ograniczaj�cych o okre�lonych proporcjach i rozmiarach~\cite{Bochkovskiy2020}. Anchor boxy s� zwykle wyznaczane na podstawie analizy danych treningowych (np. metod� k-�rednich), tak aby jak najlepiej odzwierciedla�y typowe kszta�ty i rozmiary obiekt�w wyst�puj�cych w danym zbiorze. W typowej konfiguracji YOLOv4 stosuje si� trzy anchor boxy na ka�d� kom�rk� dla ka�dej z trzech skal, co daje ��cznie dziewi�� anchor�w przypisanych do danego punktu siatki i pozwala na modelowanie obiekt�w o zr�nicowanych rozmiarach oraz proporcjach~\cite{Bochkovskiy2020}. 

\subsection*{Etap 2: Ekstrakcja cech � backbone CSPDarknet53}

Pierwszym etapem przetwarzania obrazu w YOLOv4 jest ekstrakcja cech wizualnych z wykorzystaniem g��bokiej sieci konwolucyjnej pe�ni�cej rol� \textit{backbone}. W YOLOv4 funkcj� t� realizuje architektura \textit{CSPDarknet53}, b�d�ca rozwini�ciem sieci Darknet-53 z zastosowaniem mechanizmu \textit{cross-stage partial connections}~\cite{Bochkovskiy2020}. Rozwi�zanie to pozwala na lepsze rozdzielenie przep�ywu gradient�w w sieci oraz redukcj� liczby operacji obliczeniowych bez istotnej utraty jako�ci reprezentacji cech~\cite{Bochkovskiy2020}. 

W trakcie propagacji w g��b sieci CSPDarknet53 obraz jest wielokrotnie poddawany operacjom konwolucji, normalizacji i nieliniowej aktywacji, co prowadzi do utworzenia hierarchicznej reprezentacji cech � od niskopoziomowych (kraw�dzie, tekstury) po wysokopoziomowe (struktury semantyczne). Dzi�ki temu kolejne warstwy sieci s� w stanie rozr�nia� zar�wno proste, jak i z�o�one obiekty, co jest kluczowe dla skutecznej detekcji w r�norodnych scenariuszach, w tym w z�o�onych �rodowiskach miejskich~\cite{Geron2019}. 

\subsection*{Etap 3: Agregacja wieloskalowa � SPP i PAN}

Po etapie ekstrakcji cech przez backbone wykorzystywane s� modu�y odpowiedzialne za ich dalsz� fuzj� i agregacj�. W YOLOv4 rol� t� pe�ni� mi�dzy innymi \textit{Spatial Pyramid Pooling} (SPP) oraz \textit{Path Aggregation Network} (PAN)~\cite{Bochkovskiy2020}. Modu� SPP stosuje r�wnoleg�e operacje poolingowe o r�nych rozmiarach okien, co pozwala na zwi�kszenie efektywnego pola recepcji i integracj� informacji kontekstowych z r�nych skal przestrzennych bez konieczno�ci zmiany wymiar�w wej�cia~\cite{Bochkovskiy2020}. 

Z kolei \textit{Path Aggregation Network} odpowiada za efektywn� propagacj� cech pomi�dzy warstwami ni�szymi i wy�szymi poprzez po��czenia typu top-down i bottom-up, ��cz�c informacj� semantyczn� z wy�szych poziom�w z detalami przestrzennymi z ni�szych poziom�w~\cite{Sze2017}. Taka struktura poprawia jako�� detekcji ma�ych obiekt�w oraz stabilizuje proces uczenia, poniewa� sie� otrzymuje bogatsz� i lepiej zr�nicowan� reprezentacj� cech na wszystkich poziomach rozdzielczo�ci. W efekcie mo�liwa jest jednoczesna detekcja obiekt�w o istotnie r�nych rozmiarach przy zachowaniu wysokiej precyzji lokalizacji~\cite{Bochkovskiy2020}. 

\subsection*{Etap 4: Predykcja atrybut�w obiekt�w na wielu skalach}

Na wyj�ciu modu��w agreguj�cych cechy znajduj� si� g�owy (sieci neuronowe odpowiedzialne za regresj� i klasyfikacj�) detekcyjne przypisane do trzech siatek: \(13 \times 13\), \(26 \times 26\) oraz \(52 \times 52\)~\cite{Bochkovskiy2020}. Ka�da kom�rka tych siatek generuje predykcje dla swoich anchor box�w. Dla ka�dego anchor boxa sie� przewiduje zestaw parametr�w opisuj�cych potencjalny obiekt: wsp�rz�dne przesuni�cia �rodka ramki wzgl�dem kom�rki siatki \((t_x, t_y)\), logarytmiczne przeskalowania szeroko�ci i wysoko�ci \((t_w, t_h)\), wsp�czynnik ufno�ci \(P_{obj}\) oraz wektor prawdopodobie�stw przypisania do poszczeg�lnych klas~\cite{Aggarwal2018}. 

Warto�ci wyj�ciowe s� nast�pnie przekszta�cane do przestrzeni obrazu za pomoc� funkcji nieliniowych. Wsp�rz�dne �rodka ramki s� skalowane przy u�yciu funkcji sigmoidalnej, co zapewnia ich lokalizacj� w obr�bie danej kom�rki siatki, natomiast szeroko�� i wysoko�� s� obliczane poprzez zastosowanie funkcji wyk�adniczej do przewidywanych parametr�w oraz pomno�enie przez wymiary anchor boxa~\cite{Bochkovskiy2020}. Dzi�ki temu model uczy si� jedynie wzgl�dnych modyfikacji anchor�w, co stabilizuje proces optymalizacji i poprawia zbie�no�� uczenia. 

Predykcja klas obiekt�w realizowana jest poprzez generowanie rozk�adu prawdopodobie�stwa spo�r�d wszystkich dost�pnych klas, zwykle przy u�yciu funkcji softmax lub sigmoidalnej, w zale�no�ci od przyj�tej konfiguracji~\cite{Goodfellow2016}. W przypadku YOLOv4 model jest domy�lnie trenowany na zbiorze COCO, obejmuj�cym 80 klas obiekt�w, takich jak osoby, pojazdy, zwierz�ta, elementy infrastruktury czy przedmioty codziennego u�ytku~\cite{Lin2014}. Jednocze�nie architektura pozwala na �atwe dostosowanie liczby klas do konkretnego zastosowania poprzez transfer uczenia na w�asnym zbiorze danych. 

\subsection*{Etap 5: Filtrowanie wynik�w � Non-Maximum Suppression}

Po wygenerowaniu predykcji dla wszystkich kom�rek i anchor box�w na trzech skalach powstaje bardzo du�y zbi�r potencjalnych ramek detekcyjnych. Wiele z nich odnosi si� do tego samego obiektu, r�ni�c si� nieznacznie po�o�eniem oraz warto�ci� wsp�czynnika ufno�ci. Aby otrzyma� sp�jny i nieprzepe�niony zbi�r detekcji, stosuje si� algorytm \textit{Non-Maximum Suppression} (NMS)~\cite{Geron2019,Goodfellow2016}. 

Algorytm NMS polega na iteracyjnym wybieraniu ramek o najwy�szej warto�ci ufno�ci dla danej klasy, a nast�pnie eliminowaniu tych, kt�rych miara nak�adania si� z wybran� ramk�, okre�lona wska�nikiem IoU (Intersection over Union), przekracza zadany pr�g~\cite{Geron2019}. W rezultacie pozostaj� jedynie ramki najlepiej reprezentuj�ce poszczeg�lne obiekty, co zapewnia czytelno�� i jednoznaczno�� wynik�w detekcji. Mechanizm ten jest szczeg�lnie istotny w scenach o wysokim zag�szczeniu obiekt�w, gdzie wiele anchor�w mo�e wskazywa� na ten sam element obrazu. 

\subsection*{Etap 6: Techniki treningowe i generalizacja}

YOLOv4, opr�cz zmian architektury, wykorzystuje tak�e szereg zaawansowanych technik treningowych, takich jak \textit{mosaic data augmentation}, \textit{dropblock regularization} oraz funkcja straty \textit{CIoU loss}, kt�re poprawiaj� zdolno�� modelu do generalizacji~\cite{Bochkovskiy2020}. Mosaic augmentation polega na ��czeniu fragment�w kilku obraz�w w jeden przyk�ad treningowy, co zwi�ksza r�norodno�� danych, natomiast DropBlock wprowadza losowe wyzerowywanie blok�w aktywacji, ograniczaj�c zjawisko przeuczenia~\cite{Bochkovskiy2020,Geron2019}. Z kolei CIoU loss uwzgl�dnia nie tylko nak�adanie si� ramek, ale tak�e odleg�o�� mi�dzy ich �rodkami oraz proporcje, co prowadzi do dok�adniejszej optymalizacji parametr�w ramek ograniczaj�cych~\cite{Bochkovskiy2020}. 

Zastosowanie tych metod powoduje, �e YOLOv4 osi�ga wysok� dok�adno�� predykcji przy zachowaniu kr�tkiego czasu inferencji, co czyni go szczeg�lnie atrakcyjnym w zastosowaniach czasu rzeczywistego, zw�aszcza na platformach o ograniczonych zasobach obliczeniowych. Jest to kluczowe zar�wno w systemach wbudowanych, jak i w aplikacjach wymagaj�cych przetwarzania strumieni wideo na �ywo. 

\subsection*{Etap 7: Zastosowania YOLO/YOLOv4, w tym integracja z CARLA}

Dzi�ki wysokiej szybko�ci dzia�ania i precyzyjnej detekcji obiekt�w, YOLOv4 znajduje zastosowanie w wielu dziedzinach, takich jak systemy monitoringu wizyjnego, inteligentne miasta, analiza wideo, robotyka czy systemy bezpiecze�stwa~\cite{yolo_usage}. Szczeg�lnie istotn� grup� zastosowa� stanowi� systemy wspomagania kierowcy oraz badania nad autonomiczn� jazd�, gdzie konieczne jest niezawodne wykrywanie obiekt�w w dynamicznych i z�o�onych scenach drogowych. 

Integracja YOLOv4 z symulatorem CARLA umo�liwia trenowanie i testowanie modeli detekcji w realistycznych, kontrolowanych warunkach miejskich, z uwzgl�dnieniem r�norodnych scenariuszy ruchu drogowego, zmiennych warunk�w pogodowych oraz o�wietleniowych~\cite{carla}. W takim �rodowisku model mo�e wykrywa� kluczowe obiekty infrastruktury drogowej, mi�dzy innymi pojazdy osobowe i ci�arowe, pieszych, rowerzyst�w, sygnalizacj� �wietln� oraz znaki drogowe, a tak�e inne pojazdy autonomiczne~\cite{carla,yolo_usage}. Pozwala to nie tylko oceni� skuteczno�� samej detekcji, lecz tak�e testowa� algorytmy podejmowania decyzji, planowania trajektorii oraz unikania kolizji, co ma kluczowe znaczenie w rozwoju system�w autonomicznej jazdy. 

\subsection{Architektura YOLOv4}

YOLOv4 jest jedn� z najnowszych wersji modelu YOLO, kt�ry jest jednym z najpopularniejszych algorytm�w do detekcji obiekt�w w obrazach. Jego architektura opiera si� na trzech g��wnych komponentach:
\begin{itemize}
	\item \textbf{Backbone} - jest odpowiedzialny za ekstrakcj� cech z obrazu. W YOLOv4 wykorzystano zaawansowan� sie� ResNet-50, kt�ra pozwala na szybkie i dok�adne przetwarzanie obrazu.
	\item \textbf{Neck} - ��czy cechy z r�nych warstw backbone i pomaga w ich dalszym przetwarzaniu, umo�liwiaj�c detekcj� obiekt�w w r�nych skalach. W YOLOv4 zastosowano PANet (Path Aggregation Network), kt�re poprawia reprezentacj� cech.
	\item \textbf{Head} - dokonuje finalnej klasyfikacji oraz lokalizacji obiekt�w na obrazie, przy pomocy detekcji box�w i klasyfikacji dla ka�dego wykrytego obiektu.
\end{itemize}

YOLOv4 korzysta z zaawansowanych technik, takich jak:
\begin{itemize}
	\item \textbf{DropBlock} - technika regularizacji, kt�ra pomaga zapobiega� przeuczeniu (overfitting).
	\item \textbf{CSPDarknet53} - nowoczesna sie�, kt�ra stanowi podstaw� (backbone) YOLOv4.
	\item \textbf{CIoU Loss} - funkcja straty, kt�ra poprawia dok�adno�� lokalizacji obiekt�w.
\end{itemize}

\begin{figure}[H]
	\centering
	\includegraphics[width=12cm]{Rysunki/Rozdzial3/yolov4_architecture.png}
	\caption{Architektura YOLOv4. Zawiera backbone, neck i head~\cite{yolo_usage}.}
	\label{fig:yolov4_arch}
\end{figure}

Wszystkie te elementy wsp�pracuj�, aby umo�liwi� YOLOv4 wykrywanie obiekt�w na obrazach w czasie rzeczywistym, przy zachowaniu wysokiej dok�adno�ci i wydajno�ci. Dzi�ki zastosowaniu wielu technik optymalizacyjnych, YOLOv4 osi�ga bardzo wysok� dok�adno�� i du�� szybko�� dzia�ania w por�wnaniu z poprzednimi wersjami.

\section{Zalety, wydajno�� i zastosowania algorytmu YOLOv4}

YOLOv4 (You Only Look Once version 4) to jedna z najnowocze�niejszych i najbardziej zaawansowanych wersji sieci neuronowych przeznaczonych do detekcji obiekt�w w czasie rzeczywistym. Algorytm ten wyr�nia si� znakomit� r�wnowag� pomi�dzy szybko�ci� dzia�ania a jako�ci� detekcji, co sprawia, �e z powodzeniem znajduje zastosowanie zar�wno w badaniach naukowych, jak i w aplikacjach przemys�owych, militarnych czy cywilnych.

\subsection*{Zalety i efektywno�� dzia�ania}

YOLOv4 ��czy w sobie liczne usprawnienia architektoniczne i techniczne wzgl�dem wcze�niejszych wersji (YOLOv1�YOLOv3) oraz konkurencyjnych metod takich jak Faster R-CNN czy SSD. Do jego najistotniejszych zalet nale��:

\begin{itemize}
	\item \textbf{Wysoka pr�dko�� dzia�ania} � osi�ga nawet do 65 klatek na sekund� (FPS) na wydajnych procesorach graficznych, co pozwala na detekcj� w czasie rzeczywistym \cite{Bochkovskiy2020}.
	\item \textbf{Obs�uga urz�dze� brzegowych} � dzi�ki optymalizacji sieci i wsparciu dla technologii takich jak TensorRT, YOLOv4 mo�e dzia�a� na urz�dzeniach o ograniczonej mocy obliczeniowej (np. Nvidia Jetson).
	\item \textbf{Wysoka dok�adno�� detekcji} � osi�ga wynik mAP (mean Average Precision) rz�du 43,5\% na zestawie danych COCO, co czyni go jednym z lider�w w�r�d modeli jednoetapowych (ang. \textit{one-stage detectors}) \cite{Bochkovskiy2020}.
	\item \textbf{Elastyczno�� i adaptowalno��} � model mo�na z �atwo�ci� dostosowa� do nowych klas obiekt�w poprzez proces tzw. \textit{fine-tuningu}.
	\item \textbf{Odporno�� na zak��cenia} � YOLOv4 radzi sobie z trudnymi warunkami detekcji, takimi jak cz�ciowe zas�oni�cia, rotacje, zmienne o�wietlenie czy szum.
\end{itemize}

\subsection*{Inne przyk�ady zastosowania algorytmu YOLO}

Model ten znajduje zastosowanie w szerokim zakresie dziedzin, m.in.:

\begin{itemize}
	\item \textbf{Monitorowanie i analiza ruchu drogowego} � YOLOv4 skutecznie identyfikuje pojazdy, pieszych, rowerzyst�w, znaki drogowe i inne elementy infrastruktury drogowej, umo�liwiaj�c zastosowanie w systemach ITS (Intelligent Transportation Systems) \cite{yolo_usage}.
	\item \textbf{Systemy bezpiecze�stwa i nadzoru wideo} � detekcja os�b, baga�y, podejrzanych zachowa� czy narusze� przestrzeni publicznych.
	\item \textbf{Automatyka przemys�owa} � wykrywanie defekt�w, klasyfikacja produkt�w oraz inspekcja wizualna na liniach produkcyjnych.
	\item \textbf{Robotyka i pojazdy autonomiczne} � rozpoznawanie przeszk�d i element�w otoczenia w czasie rzeczywistym w celu bezpiecznej nawigacji.
\end{itemize}

\section{Instalacja systemu YOLO}
\subsection*{Instalacja wymaganych pakiet�w}

Pierwszym krokiem w procesie instalacji jest zainstalowanie wszystkich wymaganych pakiet�w, w tym Git, Python oraz bibliotek zwi�zanych z CUDA i OpenCV. W celu zaktualizowania listy pakiet�w nale�y wykona� poni�sze polecenie:

\begin{lstlisting}[caption={Aktualizacja listy pakiet�w}]
	sudo apt update
\end{lstlisting}

Nast�pnie zainstalowane zostan� wymagane pakiety:

\begin{lstlisting}[caption={Instalacja wymaganych pakiet�w}]
	sudo apt install build-essential cmake git pkg-config libjpeg8-dev \
	libtiff5-dev libjasper-dev libpng12-dev libopencv-dev libeigen3-dev \
	libatlas-base-dev gfortran python3-dev python3-pip python3-numpy \
	libhdf5-dev libhdf5-serial-dev libprotobuf-dev protobuf-compiler \
	libgflags-dev libgoogle-glog-dev liblmdb-dev
\end{lstlisting}

\subsection*{Instalacja CUDA i cuDNN}

W przypadku ch�ci korzystania z przyspieszenia GPU, konieczna jest instalacja CUDA oraz cuDNN.

Ay zainstalowa� odpowiedni� wersj� CUDA, zgodn� z systemem, nale�y pobra� j� ze strony NVIDIA: \url{https://developer.nvidia.com/cuda-downloads}. W celu zainstalowania CUDA nale�y wykona� poni�sze polecenie:

\begin{lstlisting}[caption={Instalacja CUDA}]
	sudo apt install nvidia-cuda-toolkit
\end{lstlisting}

Aby zainstalowa� cuDNN, kt�ry jest niezb�dny do przyspieszenia oblicze� na GPU, nale�y wykona� poni�sze polecenie:

\begin{lstlisting}[caption={Instalacja cuDNN}]
	sudo apt install libcudnn7 libcudnn7-dev
\end{lstlisting}

\subsection*{Klonowanie repozytorium YOLOv4 i kompilacja}

Aby pobra� kod �r�d�owy YOLOv4, oficjalne repozytorium z GitHub jest klonowane przy u�yciu poni�szego polecenia:

\begin{lstlisting}[caption={Klonowanie repozytorium YOLOv4}]
	git clone https://github.com/AlexeyAB/darknet
	cd darknet
\end{lstlisting}

Nast�pnie plik \texttt{Makefile} nale�y edytowa�, aby w��czy� obs�ug� CUDA (GPU) oraz OpenCV, zmieniaj�c odpowiednie opcje na:

\begin{lstlisting}[caption={Edytowanie pliku Makefile}]
	GPU=1
	CUDNN=1
	OPENCV=1
\end{lstlisting}

Po dokonaniu zmian, projekt jest kompilowany za pomoc� polecenia:

\begin{lstlisting}[caption={Kompilacja projektu YOLOv4}]
	make
\end{lstlisting}

\subsection{Testowanie instalacji}

Po zako�czeniu kompilacji system mo�e zosta� przetestowany, aby upewni� si�, �e instalacja przebieg�a pomy�lnie. YOLOv4 mo�e zosta� uruchomiony na przyk�adowym obrazie przy u�yciu poni�szego polecenia:

\begin{lstlisting}[caption={Testowanie YOLOv4}]
	./darknet detector test cfg/coco.data cfg/yolov4.cfg yolov4.weights data/dog.jpg
\end{lstlisting}

Je�li wszystko zosta�o poprawnie zainstalowane, powinien zosta� wy�wietlony wynik wykrywania obiekt�w na obrazie.

\begin{figure}[H]
	\centering
	\includegraphics[width=12cm]{Rysunki/Rozdzial3/yolo_test.png}
	\label{fig:yoloTest}
	\caption[Yolo Test]{Obraz przedstawiaj�cy poprawne zainstalowanie i uruchomienie YOLOv4.}
\end{figure}

\section{Integracja detektora YOLOv4 z symulatorem CARLA}

W celu rozszerzenia funkcjonalno�ci symulatora CARLA o mo�liwo�� detekcji obiekt�w w czasie rzeczywistym, dokonano integracji modelu YOLOv4 z plikiem \texttt{manual\_control.py}. Model YOLOv4 (You Only Look Once) to zaawansowany algorytm detekcji obiekt�w w obrazie, kt�ry umo�liwia identyfikacj� oraz lokalizacj� wielu klas obiekt�w w pojedynczym przebiegu sieci neuronowej.

Proces integracji rozpocz�to od zaimportowania wymaganych bibliotek i konfiguracji �rodowiska TensorFlow:

\begin{lstlisting}[style=pythonColor, emph={import tensorflow, InteractiveSession, filter\_boxes, cfg}, caption={Importowanie bibliotek dla integracji z YOLOv4}]
	import tensorflow as tf
	from tensorflow.compat.v1 import InteractiveSession
	from core.yolov4 import filter_boxes
	from core.config import cfg
\end{lstlisting}

Zdefiniowana zosta�a r�wnie� globalna funkcja \verb|spawn_actor()|, kt�rej celem jest wczytywanie i przekszta�canie listy wsp�rz�dnych tzw. \verb|anchor boxes|, b�d�cych podstaw� w modelu YOLO do przewidywania po�o�enia obiekt�w w obrazie. Przyjmuje jako argument list� wsp�rz�dnych opisuj�cych wymiary anchor�w, nast�pnie przekszta�ca j� do struktury tr�jwymiarowej, umo�liwiaj�cej przypisanie anchor�w do trzech skal detekcji, z kt�rych ka�da operuje na prostok�tnych propozycjach. Dzi�ki takiej organizacji danych mo�liwe jest skuteczne dopasowanie anchor�w do charakterystyki obiekt�w wyst�puj�cych w obrazie, co znacz�co wp�ywa na jako�� predykcji oraz efektywno�� dzia�ania algorytmu detekcji.

\begin{lstlisting}[language=Python, style=pythonColor, emph={get\_anchors, array, reshape}, caption={Globalna funkcja \texttt{get\_anchors()} dla YOLOv4}]
	def get_anchors(anchors_path):
	anchors = np.array(anchors_path)
	return anchors.reshape(3, 3, 2)
\end{lstlisting}

Po wczytaniu obrazu z symulatora przy u�yciu biblioteki \texttt{pygame}, ramka obrazu jest skalowana i przekszta�cana na odpowiedni format:

\begin{lstlisting}[style=pythonColor, emph={pygame.surfarray.array3d, cv2.resize, np.newaxis}, caption={Wczytywanie i przetwarzanie obrazu z symulatora CARLA}]
	frame = pygame.surfarray.array3d(display)
	image_data = cv2.resize(frame, (self.input_size, self.input_size))
	image_data = image_data / 255.
	image_data = image_data[np.newaxis, ...].astype(np.float32)
\end{lstlisting}

Obraz ten trafia nast�pnie do sieci neuronowej YOLOv4, kt�ra zwraca ramki ograniczaj�ce (ang. bounding boxes), prawdopodobie�stwa detekcji oraz klasy obiekt�w. Przyk�adowo, wykorzystano funkcj� \texttt{combined\_non\_max\_suppression} do eliminacji powtarzaj�cych si� wykry�:

\begin{lstlisting}[style=pythonColor, emph={tf.image.combined\_non\_max\_suppression}, caption={Wykorzystanie funkcji \texttt{combined\_non\_max\_suppression} w celu eliminacji powt�rze�}]
	boxes, scores, classes, valid_detections = tf.image.combined_non_max_suppression(...)
\end{lstlisting}

Tak przygotowane dane s� nast�pnie przetwarzane i wizualizowane w �rodowisku symulatora CARLA.

\section{Schemat ewaluacji offline}

Opr�cz test�w on-line, w kt�rych YOLOv4 dzia�a� bezpo�rednio w symulatorze CARLA, opracowano r�wnie� schemat ewaluacji offline. Jego celem by�o szczeg�owe por�wnanie rezultat�w detekcji sieci z danymi referencyjnymi (\textit{ground truth}) generowanymi przez symulator, z wykorzystaniem metryki Intersection over Union (IoU) \cite{carla,visoiou}. 

Proponowana procedura sk�ada si� z kilku kolejnych etap�w.
\begin{enumerate}
	\item \textbf{Generowanie danych referencyjnych w CARLA.}  
	Podczas symulacji rejestrowane s� klatki z kamery RGB zamontowanej na ego�poje�dzie oraz odpowiadaj�ce im informacje o aktorach w scenie. Skrypt \texttt{bounding\_boxes.py} odczytuje struktur� \texttt{carla.BoundingBox} dla ka�dego pojazdu, przelicza wierzcho�ki bry�y 3D do uk�adu wsp�rz�dnych kamery i rzutuje je na p�aszczyzn� obrazu, wyznaczaj�c ostatecznie prostok�tne obramowanie 2D (bbox 2D). Dla ka�dej klatki zapisywany jest plik JSON zawieraj�cy list� obiekt�w z nazw� klasy, wsp�rz�dnymi prostok�ta w pikselach oraz identyfikatorem klatki/obrazu, kt�ry pozwala jednoznacznie powi�za� anotacje z konkretnym plikiem JPG. 
	
	\item \textbf{Detekcja obiekt�w za pomoc� YOLOv4.}  
	W drugim kroku YOLOv4, zaimplementowane w bibliotece Ultralytics, jest uruchamiane w trybie wsadowym na zestawie obraz�w zapisanych wcze�niej z kamery symulatora. Dla ka�dego pliku JPG model zwraca list� wykrytych obiekt�w wraz z klas�, wsp�czynnikiem pewno�ci (ang. \textit{confidence}) oraz wsp�rz�dnymi proponowanego bounding boxa 2D. Wyniki s� eksportowane do osobnego pliku JSON, w kt�rym ka�da detekcja zawiera: nazw� obrazu, nazw� klasy, wsp�rz�dne prostok�ta i warto�� confidence \cite{ultralytics_docs,ultralytics_python}. 
	
	\item \textbf{Parowanie detekcji z anotacjami CARLA.}  
	Skrypt ewaluacyjny wczytuje r�wnolegle pliki JSON z anotacjami CARLA oraz pliki z predykcjami YOLOv4. Dla ka�dej klatki obrazu wyszukiwane s� pary prostok�t�w: jeden pochodz�cy z CARLA (ground truth) oraz drugi wygenerowany przez YOLOv4, nale��ce do tej samej klasy obiektu (np. \textit{car}). Nast�pnie dla ka�dej mo�liwej pary obliczana jest warto�� IoU, a dopasowanie wybierane jest na podstawie najwi�kszej warto�ci IoU powy�ej zadanego progu (np. 0{,}5). Pozwala to zapobiec sytuacjom, w kt�rych wiele predykcji przypisano by do tego samego obiektu. 
	
	\item \textbf{Obliczanie metryk jako�ciowych.}  
	Na etapie ko�cowym skrypt zlicza poprawne dopasowania (TP � \textit{true positive}), pomini�te obiekty (FN � \textit{false negative}) oraz nadmiarowe detekcje YOLOv4 (FP � \textit{false positive}). Na tej podstawie wyznaczane s� statystyki jako�ciowe, takie jak �rednia warto�� IoU, precyzja (precision) czy czu�o�� (recall) dla wybranych scen i warunk�w pogodowych. Dodatkowo zapisywane s� rozk�ady IoU (np. histogramy), co pozwala zidentyfikowa� przypadki skrajne � bardzo dobre oraz bardzo s�abe dopasowania. Wyniki te stanowi� podstaw� do analizy wp�ywu warunk�w o�wietleniowych i atmosferycznych na dok�adno�� detekcji YOLOv4. 
\end{enumerate}

Zaproponowany schemat ewaluacji offline umo�liwia powtarzaln�, zautomatyzowan� ocen� jako�ci dzia�ania modelu YOLOv4 na danych generowanych w symulatorze CARLA. Oddzielenie etapu generowania klatek od etapu detekcji i ewaluacji u�atwia r�wnie� dalsze eksperymentowanie, np. z innymi wersjami modelu YOLO, dodatkowymi klasami obiekt�w czy zmodyfikowanymi ustawieniami kamery, bez konieczno�ci ponownego wykonywania kosztownych symulacji. 

\chapter{Wyniki bada� eksperymentalnych}

\section{Eksperyment on-line - wydajno�� systemu}
\subsection{Opis scenariusza i przebiegu test�w}
W trakcie przeprowadzania test�w w symulatorze CARLA,  dzia�anie programu zosta�o sprawdzone poprzez wykonanie serii przejazd�w po okre�lonej trasie z widoku pierwszej osoby. Ka�dy przejazd by� nagrywany a nast�pnie w trzech uprzednio wybranych miejscach wykonywany by� zrzut ekranu. Testy obejmowa�y wszelkie mo�liwe kombinacje nast�puj�cych scenariuszy:

\begin{itemize}
	\item \textbf{Przejazdy dla r�nych p�r dnia:} testy przeprowadzono zar�wno w ci�gu dnia, jak i w nocy, aby oceni� wp�yw o�wietlenia na przebieg symulacji.
	\item \textbf{Przejazdy dla r�nych warunk�w pogodowych:} badania obejmowa�y symulacje w r�nych warunkach atmosferycznych, takich jak s�o�ce oraz deszcz, co pozwoli�o na sprawdzenie, jak zmienia si� zachowanie pojazdu i wizualizacja symulacji w tych warunkach.
	\item \textbf{Przejazdy przy r�nych poziomach nasycenia ruchu:} testy wykonywano przy r�nych poziomach nat�enia ruchu samochodowego i pieszego (ma�y, �redni, du�y), aby oceni�, jak system radzi sobie w r�nych scenariuszach nat�enia ruchu.
	\item \textbf{Przejazdy dla r�nych modeli:} badania zawiera�y r�wnie� przetestowanie dla du�ego oraz ma�ego modelu YOLO.
\end{itemize}

Podczas testowania serwer uruchomiony by� na GPU, natomiast klient a tym samym program w�a�ciwy na CPU. Wynika to z faktu, i� zasoby karty graficznej by�y zaj�te przez symulator CARLA, przez co nie by�o mo�liwo�ci uruchomienia r�wnocze�nie klienta wraz z YOLO.

Poni�ej przedstawiono trzy punkty, w kt�rych dokonywane by�y pomiary skuteczno�ci funkcjonowania programu podczas jazdy. Na jego podstawie by�a sczytywana liczba klatek na sekund� FPS (ang. Frames Per Second):

\begin{enumerate}
	\item \textbf{Obraz nr 1} wykonywany by� z widocznym znakiem STOP, a tak�e samochodem stoj�cym za skrzy�owaniem. W tym przypadku podczas s�onecznego dnia o ma�ym nat�eniu ruchu dla du�ego modelu:
	\begin{figure}[H]
		\centering
		\includegraphics[width=12cm]{Rysunki/Rozdzial4/sun10car20ppl1.png}
		\caption[sun10car20ppl1]{Miejsce wykonywania obrazu nr 1.}
		\label{fig:sun10car20ppl1}
	\end{figure} 
	
	\item \textbf{Obraz nr 2} wykonywany by� z widocznymi motorami oraz samochodami na zakr�cie na chodniku. W tym przypadku podczas bezchmurnej nocy o �rednim nat�eniu ruchu dla modelu du�ego:
	\begin{figure}[H]
		\centering
		\includegraphics[width=12cm]{Rysunki/Rozdzial4/night20car40ppl2.png}
		\caption[night20car40ppl2]{Miejsce wykonywania obrazu nr 2.}
		\label{fig:night20car40ppl2}
	\end{figure} 
	
	\item \textbf{Obraz nr 3} wykonywany by� w miejscu stanowi�cym wyzwanie dla YOLO, poniewa� by�o to ruchliwe skrzy�owanie z sygnalizacj� �wietln�, samochodami oraz pieszymi. Poni�szy rysunek przedstawia scenariusz podczas deszczowego dnia o ma�ym nat�eniu ruchu dla du�ego modelu:
	\begin{figure}[H]
		\centering
		\includegraphics[width=12cm]{Rysunki/Rozdzial4/rainsun10car20ppl3.png}
		\caption[rainsun10car20ppl3]{Miejsce wykonywania obrazu nr 3.}
		\label{fig:rainsun10car20ppl3}
	\end{figure} 
\end{enumerate}

\subsection{�rodowisko sprz�towe i konfiguracja}

Eksperymenty przeprowadzono na komputerze o nast�puj�cej konfiguracji sprz�towej:
\begin{itemize}
	\item procesor: Intel Xeon E5-2697v2 (12 rdzeni, 24 w�tki),
	\item pami�� RAM: 32 GB,
	\item dysk: 500 GB,
	\item karta graficzna: NVIDIA RTX 3060 Ti (8 GB VRAM),
	\item system operacyjny: Ubuntu 18.04.
\end{itemize}

Filmy z przejazd�w nagrywano za pomoc� programu \texttt{OBS Studio} i zapisywano w formacie \texttt{.mkv}, natomiast zrzuty ekranu z symulatora przechowywano w formacie \texttt{.png}. Testy wykonano dla dw�ch modeli detekcji: ma�ego \texttt{yolov4-tiny-416} oraz du�ego \texttt{yolov4-416}. Dla ka�dego scenariusza rejestrowano warto�ci \textit{FPS}, co umo�liwi�o ocen� wydajno�ci systemu w r�nych warunkach symulacji.

Testy by�y przeprowadzane r�wnie� dla modeli o r�nym stopniu skomplikowania, w tym modelu ma�ego \texttt{yolov4-tiny-416} oraz modelu du�ego \texttt{yolov4-416}, z rejestrowaniem warto�ci \textit{FPS} w ka�dym przypadku, co pozwoli�o na ocen� wydajno�ci systemu w r�nych warunkach symulacji.

\subsection{Analiza wynik�w wydajno�ciowych}

Na podstawie tabel \ref{tab:wyniki_model_duzy} oraz \ref{tab:wyniki_model_maly} mo�na zauwa�y�, �e liczba klatek na sekund� po stronie serwera jest bardzo podobna dla obu modeli detekcji. W warunkach dziennych przy s�onecznej pogodzie i ma�ym nat�eniu ruchu serwer osi�ga typowo 20--25 FPS, zar�wno dla modelu \texttt{yolov4-416}, jak i \texttt{yolov4-tiny-416}. Przy wzro�cie liczby pojazd�w i pieszych do poziomu �redniego oraz du�ego warto�ci te stopniowo spadaj� do oko�o 14--19 FPS, co wida� zar�wno w scenariuszach �dzie�, s�o�ce�, jak i �dzie�, deszcz�. Oznacza to, �e w tych konfiguracjach g��wnym czynnikiem wp�ywaj�cym na wydajno�� serwera jest z�o�ono�� sceny w symulatorze CARLA, a nie rozmiar zastosowanego modelu detekcji.

Najwi�ksze r�nice mi�dzy poszczeg�lnymi scenariuszami pogodowymi i porami dnia pojawiaj� si� w przypadku scen nocnych oraz przy intensywnych opadach deszczu. Dla konfiguracji �noc, czyste niebo� oraz �noc, deszcz� warto�ci FPS na serwerze spadaj� do poziomu oko�o 6--11 FPS niezale�nie od tego, czy wykorzystywany jest model du�y, czy ma�y. Wynika to z wi�kszej z�o�ono�ci o�wietlenia, licznych �r�de� �wiat�a sztucznego oraz dodatkowych efekt�w pogodowych, takich jak odbicia na mokrej nawierzchni. Symulator musi w�wczas przetwarza� bardziej wymagaj�c� scen� 3D, co ogranicza liczb� generowanych klatek i w praktyce �dominuje� koszty obliczeniowe wzgl�dem samej detekcji obiekt�w.

Wyra�ne r�nice mi�dzy modelami widoczne s� natomiast po stronie klienta. Dla modelu \texttt{yolov4-416} cz�� kliencka, odpowiedzialna za wizualizacj� i logik� sterowania, pracuje zwykle z pr�dko�ci� oko�o 4--8 FPS we wszystkich analizowanych scenariuszach. W przypadku modelu \texttt{yolov4-tiny-416} warto�ci te rosn� do przedzia�u 11--16 FPS, przy czym najwy�sze wyniki osi�gane s� w dzie� przy ma�ym nat�eniu ruchu, a najni�sze w nocy przy du�ej liczbie aktor�w. Pokazuje to, �e uproszczony model detekcji znacz�co zmniejsza obci��enie CPU oraz koszty przetwarzania danych po stronie klienta, co przek�ada si� na bardziej p�ynne dzia�anie interfejsu.

Podsumowuj�c, wyniki eksperymentu on-line wskazuj�, �e w badanej konfiguracji sprz�towej w�skim gard�em wydajno�ci jest g��wnie symulator CARLA uruchomiony na GPU, kt�ry ogranicza liczb� FPS na serwerze do warto�ci rz�du 6--25 w zale�no�ci od warunk�w pogodowych i nat�enia ruchu. Zastosowanie modelu \texttt{yolov4-tiny-416} poprawia natomiast odczuwaln� p�ynno�� pracy po stronie klienta, umo�liwiaj�c wygodniejsze testowanie systemu w czasie zbli�onym do rzeczywistego, zw�aszcza w scenach dziennych o mniejszej z�o�ono�ci.

\begin{table}[H]
	\centering
	\adjustbox{max width=\textwidth}{  % Automatyczne dopasowanie szeroko�ci tabeli do szeroko�ci strony
		\renewcommand{\arraystretch}{1.3}  % Zwi�kszenie odst�p�w w tabeli
		\begin{tabular}{|c|l|c|c|c|}
			\hline
			\multirow{2}{*}{\textbf{Lp.}} & \multirow{2}{*}{\textbf{Scenariusz testowy}} & \multicolumn{3}{c|}{\textbf{Obrazy [FPS]}} \\
			\cline{3-5}
			& & \textbf{Nr 1 Serwer/ Klient} & \textbf{Nr 2 Serwer/ Klient} & \textbf{Nr 3 Serwer/ Klient} \\  
			\hline
			1.  & Dzie�, S�o�ce, Ma�y ruch  &  20/11  &  25/12  &  23/11  \\
			\hline
			2.  & Dzie�, S�o�ce, �redni ruch  &  16/11  &  22/12  &  20/13  \\
			\hline
			3.  & Dzie�, S�o�ce, Du�y ruch  &  14/13  &  19/12  &  17/12  \\
			\hline
			4.  & Dzie�, Deszcz, Ma�y ruch  &  19/12  &  23/11  &  22/11  \\
			\hline
			5.  & Dzie�, Deszcz, �redni ruch  &  16/13  &  20/11  &  19/12  \\
			\hline
			6.  & Dzie�, Deszcz, Du�y ruch  &  14/14  &  18/12  &  17/12  \\
			\hline
			7.  & Noc, Czyste niebo, Ma�y ruch  &  7/15  &  9/12  & 9/13  \\
			\hline
			8.  & Noc, Czyste niebo, �redni ruch  &  6/14  &  7/13  &  8/12  \\
			\hline
			9.  & Noc, Czyste niebo, Du�y ruch  &  6/13  &  8/14  &  8/14  \\
			\hline
			10. & Noc, Deszcz, Ma�y ruch  &  7/14  &  9/13  &  9/12  \\
			\hline
			11. & Noc, Deszcz, �redni ruch  &  7/14  &  8/13  &  8/14  \\
			\hline
			12. & Noc, Deszcz, Du�y ruch  &  7/15  &  8/16  &  7/15  \\
			\hline
		\end{tabular}
	}
	\caption{Wyniki test�w wydajno�ciowych modelu ma�ego.}
	\label{tab:wyniki_model_maly}
\end{table}

\begin{table}[h]
	\centering
	\resizebox{\textwidth}{!}{
		\renewcommand{\arraystretch}{1.3} % Zwi�kszenie odst�p�w w tabeli
		\begin{tabular}{|c|l|c|c|c|}
			\hline
			\multirow{2}{*}{\textbf{Lp.}} & \multirow{2}{*}{\textbf{Scenariusz testowy}} & \multicolumn{3}{c|}{\textbf{Obrazy  [FPS]}} \\
			\cline{3-5}
			& & \textbf{Nr 1} \textbf{Serwer/ Klient} & \textbf{Nr 2} \textbf{Serwer/ Klient} & \textbf{Nr 3} \textbf{Serwer/ Klient} \\  
			\hline
			1.  & Dzie�, S�o�ce, Ma�y ruch  &  20/ 4  &  25/ 4  &  23/ 4  \\
			\hline
			2.  & Dzie�, S�o�ce,  �redni ruch  &  18/ 4  &  22/ 4  &  19/ 4  \\
			\hline
			3.  & Dzie�, S�o�ce,  Du�y ruch  &  14/ 4  &  15/ 4  &  16/ 4  \\
			\hline
			4.  & Dzie�, Deszcz, Ma�y ruch  &  19/ 4  &  22/ 4  &  21/ 4  \\
			\hline
			5.  & Dzie�, Deszcz, �redni ruch  &  16/ 4  &  19/ 4  &  19/ 4  \\
			\hline
			6.  & Dzie�, Deszcz, Du�y ruch  &  14/ 4  &  16/ 4  &  16/ 4  \\
			\hline
			7.  & Noc, Czyste niebo, Ma�y ruch  &  11/ 4  &  22/ 4  &  12/ 4  \\
			\hline
			8.  & Noc, Czyste niebo, �redni ruch  &  6/ 4  &  8/ 4  &  8/ 4 \\
			\hline
			9.  & Noc, Czyste niebo, Du�y ruch  &  6/ 4  &  8/ 4  &  8/ 4  \\
			\hline
			10. & Noc, Deszcz, Ma�y ruch  &  7/ 4  &  8/ 4 &  8/ 4  \\
			\hline
			11. & Noc, Deszcz, �redni ruch  &  6/ 4  &  8/ 4  &  8/ 4  \\
			\hline
			12. & Noc, Deszcz, Du�y ruch  &   6/ 4  &  8/ 4  &  8/ 4  \\
			\hline
		\end{tabular}
	}
	\caption{Wyniki test�w wydajno�ciowych dla modelu du�ego.}
	\label{tab:wyniki_model_duzy}
\end{table}

\subsection{Przyk�adowe funkcjonalno�ci oprogramowania w trybie on-line}

Zaimplementowane oprogramowanie w �rodowisku symulacyjnym CARLA, rozszerzone o integracj� z algorytmem detekcji YOLOv4, oferuje szereg funkcjonalno�ci pozwalaj�cych na efektywne testowanie i wizualizacj� system�w percepcyjnych pojazdu autonomicznego. W niniejszym rozdziale przedstawiono najwa�niejsze elementy i mechanizmy dzia�ania, kt�re sk�adaj� si� na system przetwarzania i analizy danych w czasie rzeczywistym.

\begin{enumerate}
\item \textbf{Pobieranie i przetwarzanie obrazu}

Podstaw� dzia�ania systemu detekcji jest regularne pobieranie aktualnych danych wizualnych z renderowanej sceny symulatora. Realizowane jest to poprzez bibliotek� \texttt{pygame}, kt�ra umo�liwia bezpo�redni dost�p do zawarto�ci ekranu:

\begin{lstlisting}[style=pythonColor, emph={array3d, 0swapaxes}, caption={Pobieranie i przetwarzanie obrazu w symulatorze CARLA}]
	frame = pygame.surfarray.array3d(display)
	frame = frame.swapaxes(0,1)
\end{lstlisting}

Obraz nast�pnie poddawany jest normalizacji oraz skalowaniu do rozmiaru oczekiwanego przez sie� YOLOv4 (domy�lnie 416x416 pikseli). W ten spos�b przygotowane dane wej�ciowe s� gotowe do przekazania do modelu detekcyjnego.

\item \textbf{Wykrywanie obiekt�w}

Wykorzystanie modelu YOLOv4 pozwala na detekcj� wielu klas obiekt�w w czasie rzeczywistym. Wczytany model (za pomoc� TensorFlow) analizuje dostarczony obraz i zwraca zestaw predykcji, zawieraj�cy wsp�rz�dne obiekt�w oraz ich prawdopodobie�stwo klasyfikacji:

\begin{lstlisting}[style=pythonColor, emph={infer, tf.image.combined\_non\_max\_suppression}, caption={Wykrywanie obiekt�w za pomoc� YOLOv4}]
	pred_bbox = infer(batch_data)
	...
	boxes, scores, classes, valid_detections = tf.image.combined_non_max_suppression(...)
\end{lstlisting}

System przetwarza surowe dane wyj�ciowe, filtruj�c i formatuj�c wyniki z u�yciem funkcji pomocniczych (np. \texttt{utils.format\_boxes()}). Dodatkowo odczytywane s� nazwy klas z pliku konfiguracyjnego, aby przypisa� odpowiedni� etykiet� do ka�dego wykrytego obiektu.

\item \textbf{Filtrowanie klas}

U�ytkownik mo�e zdecydowa�, kt�re klasy obiekt�w maj� by� uwzgl�dnione podczas renderowania. Domy�lnie dozwolone s� wszystkie klasy zawarte w pliku \texttt{.names}, jednak istnieje mo�liwo�� ograniczenia listy do wybranych etykiet (np. tylko \texttt{person}, \texttt{car}):

\begin{lstlisting}[style=pythonColor, emph={allowed\_classes, class\_name}, caption={Filtrowanie wykrytych klas obiekt�w}]
	allowed_classes = ['person', 'car']
	...
	if class_name not in allowed_classes:
	deleted_indx.append(i)
\end{lstlisting}

Dzi�ki temu u�ytkownik mo�e ukierunkowa� system detekcji na konkretne cele, co jest przydatne podczas testowania okre�lonych scenariuszy, np. wykrywania pieszych na przej�ciach.

\item \textbf{Wizualizacja wykrytych obiekt�w}

Wykryte obiekty s� rysowane na ekranie w postaci prostok�t�w ograniczaj�cych (ang. \textit{bounding boxes}). Ka�dy z nich zawiera r�wnie� tekstow� etykiet� klasy obiektu:

\begin{lstlisting}[style=pythonColor, emph={pygame.draw.rect, bbox\_font.render}, caption={Wizualizacja wykrytych obiekt�w na ekranie}]
	pygame.draw.rect(display, (255, 255, 255), rect, 2)
	label = bbox_font.render(classname, True, (255,255,255))
\end{lstlisting}

System umo�liwia r�wnie� dynamiczne skalowanie czcionki, co wp�ywa na czytelno�� wynik�w. Takie podej�cie znacz�co u�atwia analiz� dzia�ania modelu w czasie rzeczywistym, umo�liwiaj�c wizualn� weryfikacj� dok�adno�ci wykry�.

\item \textbf{Reakcja na dane wej�ciowe}

W oprogramowaniu zaimplementowany zosta� system obs�ugi zdarze� klawiatury, umo�liwiaj�cy r�czne sterowanie pojazdem w trybie symulacyjnym:

\begin{lstlisting}[style=pythonColor, emph={KeyboardControl, controller.parse\_events}, caption={Obs�uga zdarze� klawiatury w symulatorze}]
	controller = KeyboardControl(world, args.autopilot)
	if controller.parse_events(client, world, clock):
	return
\end{lstlisting}

Dzi�ki temu operator mo�e aktywnie testowa� system detekcji w r�nych sytuacjach � zmieniaj�c pr�dko�� pojazdu, tor jazdy czy ustawienia kamery.

\item \textbf{Integracja komponent�w w p�tli g��wnej}

Wszystkie powy�sze funkcje zosta�y zintegrowane w g��wnej p�tli gry \texttt{game\_loop()}, kt�ra odpowiada za nieprzerwan� aktualizacj� symulacji, wykrywanie obiekt�w i renderowanie wynik�w:

\begin{lstlisting}[style=pythonColor, emph={game\_loop, world.render}, caption={P�tla g��wna symulacji CARLA}]
	while True:
	clock.tick_busy_loop(60)
	...
	detections = []
	...
	world.render(display, detections)
	pygame.display.flip()
\end{lstlisting}

System dzia�a w czasie rzeczywistym z cz�stotliwo�ci� od�wie�ania obrazu oko�o 60 FPS, co czyni go u�ytecznym narz�dziem do eksperyment�w w dziedzinie autonomicznej jazdy.
\end{enumerate}

\section{Eksperyment offline -- weryfikacja poprawno�ci detekcji}

Drugim etapem bada� by� eksperyment przeprowadzony w trybie offline, kt�rego celem by�a ilo�ciowa ocena poprawno�ci dzia�ania detektora YOLOv4. W odr�nieniu od test�w on-line, w kt�rych analizowano g��wnie wydajno�� systemu mierzon� liczb� klatek na sekund�, w eksperymencie offline por�wnywano wyniki detekcji sieci z danymi referencyjnymi (\textit{ground truth}) generowanymi przez symulator CARLA na podstawie tr�jwymiarowych bry� \textit{bounding box} przypisanych do aktor�w na scenie.

Szczeg�owy schemat procedury ewaluacji offline, obejmuj�cy zapis anotacji z symulatora, uruchamianie detektora YOLOv4 w trybie wsadowym oraz obliczanie metryki Intersection over Union (IoU), zosta� opisany w podrozdziale~3.6. W niniejszym rozdziale ograniczono si� do prezentacji g��wnych za�o�e� eksperymentu oraz wybranych rezultat�w.

Do test�w wybrano reprezentatywny zestaw scen obejmuj�cych r�ne pory dnia (dzie�, noc), warunki pogodowe (s�o�ce, deszcz) oraz poziomy nat�enia ruchu (ma�y, �redni, du�y). Dla ka�dej kombinacji wygenerowano sekwencj� klatek z kamery RGB oraz odpowiadaj�ce im pliki JSON zawieraj�ce anotacje \textit{ground truth}. Nast�pnie te same obrazy zosta�y przetworzone przez model YOLOv4 w trybie offline, a skrypt ewaluacyjny obliczy� warto�ci IoU dla dopasowanych par ramek oraz zliczy� liczby detekcji poprawnych (TP), pomini�tych (FN) i fa�szywych (FP).

Uzyskane wyniki wskazuj�, �e najwy�sze warto�ci �redniego IoU oraz najwi�kszy odsetek poprawnych detekcji osi�gane s� w warunkach dziennych przy dobrym o�wietleniu i niewielkim nat�eniu ruchu. W scenariuszach nocnych oraz przy intensywnych opadach deszczu obserwuje si� obni�enie IoU oraz wzrost liczby b��dnych detekcji, co jest sp�jne z wynikami test�w wydajno�ciowych i potwierdza, �e trudne warunki �rodowiskowe wp�ywaj� zar�wno na liczb� klatek na sekund�, jak i na jako�� predykcji modelu.

\subsection{Metryka Intersection over Union (IoU)}

Do ilo�ciowej oceny poprawno�ci dzia�ania detektora wykorzystano metryk� Intersection over Union (IoU), por�wnuj�c� prostok�ty referencyjne pochodz�ce z symulatora CARLA z ramkami przewidywanymi przez sie� YOLOv4. Dla ka�dej pary ramek tej samej klasy oblicza si� stosunek pola cz�ci wsp�lnej do pola sumy obu prostok�t�w, zgodnie z zale�no�ci�
\[
\mathrm{IoU} = \frac{\mathrm{area}(B_{\text{CARLA}} \cap B_{\text{YOLO}})}{\mathrm{area}(B_{\text{CARLA}} \cup B_{\text{YOLO}})}.
\]
Warto�� IoU zawiera si� w przedziale od 0 do 1, gdzie 1 oznacza idealne pokrycie obiektu przez predykcj� modelu, natomiast warto�ci bliskie 0 �wiadcz� o du�ym przesuni�ciu lub znacznym niedopasowaniu rozmiar�w ramek. W eksperymencie przyj�to, �e detekcja jest poprawna, je�eli klasy obiektu s� zgodne, a IoU przekracza ustalony pr�g (np. 0,5), co pozwala na zliczanie liczby trafie� (TP), pomini�� (FN) oraz fa�szywych detekcji (FP) w analizowanych scenach.

\subsection{Zestaw scen i konfiguracja eksperymentu offline}

Eksperyment offline zosta� przeprowadzony na serii przejazd�w zarejestrowanych
w symulatorze CARLA, r�ni�cych si� konfiguracj� �rodowiska, warunkami
atmosferycznymi oraz z�o�ono�ci� sceny. Dla ka�dego przejazdu zapisano
sekwencj� klatek z kamery RGB skierowanej do przodu pojazdu oraz odpowiadaj�ce
im pliki JSON z anotacjami \textit{ground truth}, generowanymi przez skrypt
\texttt{bounding\_boxes.py}. Anotacje zawiera�y informacje o po�o�eniu oraz
klasie obiekt�w widocznych w obrazie, co umo�liwia�o p�niejsze por�wnanie
dzia�ania algorytmu detekcji z danymi referencyjnymi.

W trakcie generowania anotacji uwzgl�dniano wy��cznie obiekty znajduj�ce si�
w odleg�o�ci do 75 metr�w od kamery (parametr zasi�gu detekcji oznaczony jako
\texttt{d75}). Ograniczenie to wynika z typowego zasi�gu skutecznej detekcji
dla kamer stosowanych w pojazdach oraz z faktu, �e obiekty po�o�one dalej maj�
znikomy wp�yw na bie��ce decyzje uk�ad�w wspomagania kierowcy. Dodatkowo
pozwala to unikn�� bardzo ma�ych obiekt�w w dalszym planie, kt�rych poprawne
oznaczenie w anotacjach oraz detekcja przez model s� szczeg�lnie trudne.

Przejazdy realizowano na dw�ch mapach dost�pnych w symulatorze, oznaczonych
jako \textit{Town03} oraz \textit{Town04}. Pierwsza z nich reprezentuje wi�kszy
obszar miejski z wieloma skrzy�owaniami, rondem oraz rozbudowan� infrastruktur�
drogow�, natomiast druga odwzorowuje fragment drogi szybkiego ruchu poprowadzonej
poza zabudow� miejsk�, z charakterystycznym uk�adem w kszta�cie ��semki�
oraz otoczeniem le�nym.[web:77] W obu przypadkach scena by�a uzupe�niona o ruch
pojazd�w sterowanych przez wbudowany system ruchu drogowego, co pozwoli�o
uzyska� zr�nicowane rozk�ady po�o�e� i pr�dko�ci obiekt�w.

Docelowa liczba aktor�w w scenie zosta�a ustawiona na oko�o 200. Wi�kszo�� z nich
stanowi�y pojazdy (\textit{vehicles}), jednak w otoczeniu mog�y pojawia� si�
r�wnie� inni uczestnicy ruchu, tacy jak piesi czy rowerzy�ci. Nale�y podkre�li�,
�e liczba obiekt�w obecnych w pojedynczej klatce nie jest dok�adnie r�wna 200,
gdy� symulator dynamicznie do��cza i usuwa aktor�w w zale�no�ci od ich po�o�enia
wzgl�dem aktualnie aktywnego obszaru mapy oraz bie��cych warunk�w ruchu.
W praktyce przek�ada si� to na zmienn� g�sto�� ruchu, co jest korzystne z punktu
widzenia testowania algorytmu w r�nych scenariuszach � od wzgl�dnie pustej
drogi po sytuacje bardziej zat�oczone.

Dla ka�dej mapy przygotowano kilka wariant�w pogodowych i o�wietleniowych,
odpowiadaj�cych predefiniowanym ustawieniom symulatora, takim jak s�oneczny dzie�
(\textit{ClearNoon}), deszcz w ci�gu dnia (\textit{HardRainNoon}) czy noc bez
zachmurzenia (\textit{ClearNight}).[web:82] Pozwoli�o to oceni�, jak na jako��
detekcji wp�ywaj� warunki o�wietleniowe (r�wnomierne o�wietlenie w po�udnie,
silne kontrasty podczas zachodu s�o�ca, ograniczona widoczno�� i punktowe
�r�d�a �wiat�a w nocy) oraz efekty pogodowe, takie jak intensywne opady deszczu
i mokra nawierzchnia. W scenach deszczowych dodatkowym utrudnieniem s� krople
na obiektywie kamery oraz odbicia �wiate� na jezdni, kt�re mog� by� mylone przez
model z rzeczywistymi obiektami.

Na mapie \textit{Town03} rejestrowano g��wnie sceny miejskie, obejmuj�ce szersze
skrzy�owania, zabudow� oraz elementy infrastruktury drogowej, takie jak sygnalizacja
�wietlna czy przej�cia dla pieszych.[web:77] W scenariuszu dziennym, przy dobrej
pogodzie, ulice s� jasno o�wietlone, a w kadrze pojawiaj� si� pojedyncze pojazdy
oraz motocykle, co sprzyja wyra�nemu odwzorowaniu kontur�w obiekt�w i oznacze�
poziomych. W wariancie deszczowej nocy obraz staje si� znacznie trudniejszy do
analizy: o�wietlenie pochodzi g��wnie z lamp ulicznych i reflektor�w pojazd�w,
a mokra nawierzchnia powoduje liczne odbicia �wiat�a. W po��czeniu z ciemnym
t�em zabudowy prowadzi to do powstawania obszar�w prze�wietlonych oraz fragment�w
sceny pogr��onych w cieniu, co stanowi istotne utrudnienie dla algorytmu detekcji.

\begin{figure}[H]
	\centering
	\includegraphics[width=12cm]{Rysunki/Rozdzial4/town_03_cars_200_clear_noon_d75_scen_01.png}
	\caption{Scenariusz \textit{Town03, dzie�, s�o�ce} z widocznymi pojazdami na skrzy�owaniu.}
	\label{fig:town03_clear_noon_offline}
\end{figure}

\begin{figure}[H]
	\centering
	\includegraphics[width=12cm]{Rysunki/Rozdzial4/town_03_cars_200_rainy_night_d75_scen_01.png}
	\caption{Scenariusz \textit{Town03, deszczowa noc} z odbiciami �wiate� na mokrej nawierzchni.}
	\label{fig:town03_rainy_night_offline}
\end{figure}

Mapa \textit{Town04} odwzorowuje odcinek drogi szybkiego ruchu w terenie
poza miejskim, z otoczeniem w postaci lasu oraz zbiornik�w wodnych.[web:79]
Dominuj� tu d�u�sze, proste odcinki oraz �agodne �uki, a ruch odbywa si� z
wi�kszymi pr�dko�ciami ni� w przypadku scen miejskich. W scenariuszu dziennym
z intensywnymi opadami deszczu widoczno�� w dalszej cz�ci sceny ograniczaj�
krople deszczu oraz zamglenie, co wp�ywa na kontrast mi�dzy pojazdami a t�em.
Mokra nawierzchnia drogi powoduje dodatkowo powstawanie odbi�, kt�re mog�
zaburza� wizualn� separacj� pas�w ruchu oraz sylwetek pojazd�w.

W scenariuszu nocnym przy bezchmurnym niebie dominuj� punktowe �r�d�a �wiat�a,
takie jak latarnie i reflektory pojazd�w, co prowadzi do du�ych r�nic jasno�ci
pomi�dzy poszczeg�lnymi fragmentami obrazu. Cz�� obiekt�w znajduje si� w
p�mroku, a cz�� jest silnie o�wietlona, co mo�e skutkowa� utrat� detali w
jasnych obszarach oraz brakiem informacji w cieniach. Z punktu widzenia
eksperymentu offline pozwala to jednak zweryfikowa�, w jakim stopniu model
detekcji radzi sobie z takimi skrajnymi warunkami i czy zachowuje stabilno��
wynik�w w otoczeniu drogi szybkiego ruchu.

\begin{figure}[H]
	\centering
	\includegraphics[width=12cm]{Rysunki/Rozdzial4/town_04_cars_200_hard_rain_noon_d75_scen_01.png}
	\caption{Scenariusz \textit{Town04, dzie�, silny deszcz} na drodze szybkiego ruchu.}
	\label{fig:town04_hard_rain_noon_offline}
\end{figure}

\begin{figure}[H]
	\centering
	\includegraphics[width=12cm]{Rysunki/Rozdzial4/town_04_cars_200_clear_night_d75_scen_01.png}
	\caption{Scenariusz \textit{Town04, noc, czyste niebo} z punktowymi �r�d�ami �wiat�a.}
	\label{fig:town04_clear_night_offline}
\end{figure}

\subsection{Przebieg przetwarzania w eksperymencie offline}

Ca�y eksperyment offline zosta� zrealizowany jako sekwencja krok�w, w kt�rej ka�dy etap odpowiada osobnemu skryptowi. Pozwoli�o to oddzieli� proces generowania danych referencyjnych od detekcji YOLOv4 oraz od w�a�ciwej ewaluacji.

W pierwszym etapie uruchamiany jest skrypt \texttt{bounding\_boxes.py}, kt�ry podczas przejazdu w symulatorze CARLA odczytuje list� aktor�w znajduj�cych si� w scenie oraz ich struktury \texttt{carla.BoundingBox}. Na tej podstawie wyznaczane s� obrysy 2D obiekt�w w obrazie z kamery, ograniczone do zasi�gu 75 metr�w od punktu obserwacji. Dla ka�dej klatki generowany jest plik JSON zawieraj�cy anotacje \textit{ground truth}: identyfikator obrazu, klas� obiektu oraz wsp�rz�dne prostok�ta w pikselach.

W drugim etapie wykorzystywany jest skrypt \texttt{yolo\_cpu.py}, kt�ry w trybie wsadowym wczytuje zapisane wcze�niej obrazy i przetwarza je za pomoc� modelu YOLOv4. Dla ka�dej klatki zapisywana jest lista wykrytych obiekt�w z podan� klas�, ramk� 2D oraz wsp�czynnikiem pewno�ci detekcji. Wyniki te trafiaj� do osobnych plik�w JSON, co umo�liwia ich p�niejsz� analiz� niezale�nie od dzia�ania symulatora.

Trzeci etap stanowi w�a�ciwa ewaluacja jako�ci detekcji, realizowana przez skrypt \texttt{evaluate\_iou.py}. Program ten wczytuje pary plik�w JSON z anotacjami CARLA oraz z predykcjami YOLOv4, dopasowuje ramki tej samej klasy na podstawie maksymalnej warto�ci IoU powy�ej ustalonego progu i oblicza statystyki jako�ciowe. Dla ka�dego przejazdu wyznaczane s� m.in. �rednie warto�ci IoU, liczba trafie� (TP), fa�szywych detekcji (FP) oraz pomini�� obiekt�w (FN).

Dodatkowo, pomocniczy skrypt \texttt{bbox\_image.py} umo�liwia wizualn� weryfikacj� wynik�w. Na podstawie wybranych plik�w JSON rysuje on ramki pochodz�ce z CARLA lub z YOLOv4 na zapisanych obrazach, co pozwala na �atwe wyszukanie klatek z bardzo dobrym, przeci�tnym oraz s�abym dopasowaniem i zilustrowanie ich w dalszej cz�ci rozdzia�u.

\begin{table}[H]
	\centering
	\resizebox{\textwidth}{!}{
		\renewcommand{\arraystretch}{1.3}
		\begin{tabular}{|l|c|c|}
			\hline
			\textbf{Scenariusz} & \textbf{Liczba klatek} & \textbf{�rednie IoU} \\
			\hline
			Town03, dzie�, s�o�ce, aktorzy $\approx$ 200       & 378 & 0{,}76 \\
			\hline
			Town03, noc, czyste niebo, aktorzy $\approx$ 200   & 333 & 0{,}69 \\
			\hline
			Town03, zach�d s�o�ca, aktorzy $\approx$ 200       & 223 & 0{,}72 \\
			\hline
			Town03, zach�d s�o�ca, aktorzy $\approx$ 200       & 445 & 0{,}71 \\
			\hline
			Town03, dzie�, silny deszcz, aktorzy $\approx$ 200 & 400 & 0{,}64 \\
			\hline
			Town03, deszczowa noc, aktorzy $\approx$ 200       & 429 & 0{,}58 \\
			\hline
			Town04, noc, czyste niebo, aktorzy $\approx$ 200   & 262 & 0{,}68 \\
			\hline
			Town04, dzie�, s�o�ce, aktorzy $\approx$ 200       & 326 & 0{,}75 \\
			\hline
			Town04, dzie�, s�o�ce, aktorzy $\approx$ 200       & 302 & 0{,}74 \\
			\hline
			Town04, zach�d s�o�ca, aktorzy $\approx$ 200       & 336 & 0{,}70 \\
			\hline
			Town04, dzie�, silny deszcz, aktorzy $\approx$ 200 & 286 & 0{,}63 \\
			\hline
			Town04, deszczowa noc, aktorzy $\approx$ 200       & 408 & 0{,}56 \\
			\hline
		\end{tabular}
	}
	\caption{Przyk�adowe wyniki eksperymentu offline dla wszystkich zarejestrowanych scen symulacji CARLA.}
	\label{tab:iou_scenarios}
\end{table}

\subsection{Analiza wynik�w eksperymentu offline}

Zestawienie wynik�w w tab.~\ref{tab:iou_scenarios} pokazuje, �e warto�ci �redniego IoU silnie zale�� od warunk�w o�wietleniowych i pogodowych. Poni�ej przedstawiono przyk�adowe klatki z na�o�onymi ramkami \textit{ground truth} pochodz�cymi z symulatora CARLA oraz detekcjami sieci YOLO dla dw�ch scenariuszy, w kt�rych model radzi sobie wyra�nie s�abiej ni� w s�oneczny dzie�.

\begin{figure}[H]
	\centering
	\includegraphics[width=12cm]{Rysunki/Rozdzial4/town03_clear_night_carla.png}
	\caption{Scenariusz \textit{Town03, noc, czyste niebo} -- ramki \textit{ground truth} CARLA dla pojazd�w widocznych w kadrze.}
	\label{fig:town03_clear_night_carla}
\end{figure}

\begin{figure}[H]
	\centering
	\includegraphics[width=12cm]{Rysunki/Rozdzial4/town03_clear_night_yolo.png}
	\caption{Scenariusz \textit{Town03, noc, czyste niebo} -- detekcje YOLO dla tej samej klatki.}
	\label{fig:town03_clear_night_yolo}
\end{figure}

W scenariuszu \textit{Town03, clear night} widoczna jest wyra�na r�nica pomi�dzy liczb� obiekt�w oznaczonych w anotacjach CARLA a liczb� ramek wygenerowanych przez YOLO. Cz�� pojazd�w pozostaje niewykryta, poniewa� ich sylwetki s� zbyt ciemne lub znajduj� si� w znacznej odleg�o�ci od kamery, przez co zajmuj� jedynie kilka pikseli w obrazie. Powoduje to spadek �redniego IoU oraz wzrost liczby pomini�� (FN), mimo relatywnie niewielkiej liczby fa�szywych detekcji.

\begin{figure}[H]
	\centering
	\includegraphics[width=12cm]{Rysunki/Rozdzial4/town03_hard_rain_noon_carla.png}
	\caption{Scenariusz \textit{Town03, dzie�, silny deszcz} -- ramki \textit{ground truth} CARLA na autostradzie.}
	\label{fig:town03_hard_rain_noon_carla}
\end{figure}

\begin{figure}[H]
	\centering
	\includegraphics[width=12cm]{Rysunki/Rozdzial4/town03_hard_rain_noon_yolo.png}
	\caption{Scenariusz \textit{Town03, dzie�, silny deszcz} -- detekcje YOLO dla tej samej klatki.}
	\label{fig:town03_hard_rain_noon_yolo}
\end{figure}

Drugi przyk�ad pochodzi z przejazdu po odcinku drogi szybkiego ruchu w warunkach \textit{Town03, hard rain noon}. Pomimo dziennej pory i lepszego o�wietlenia ni� w scenie nocnej, intensywne opady deszczu obni�aj� kontrast pomi�dzy pojazdami a t�em oraz cz�ciowo rozmywaj� ich kontury. YOLO nadal nie odtwarza wszystkich obiekt�w z anotacji CARLA, cho� wykrywa ich wi�cej ni� w scenie nocnej � szczeg�lnie dobrze rozpoznawane s� pojazdy znajduj�ce si� bli�ej kamery, natomiast pomini�cia dotycz� g��wnie obiekt�w dalszych oraz cz�ciowo zas�oni�tych.

\paragraph{Podsumowanie.}
Przedstawione przyk�ady potwierdzaj�, �e najwi�ksze problemy detektor napotyka w sytuacjach, gdy informacja wizualna o obiekcie jest ograniczona przez s�abe o�wietlenie, intensywne opady lub du�� odleg�o�� od kamery. Z punktu widzenia dalszego rozwoju systemu oznacza to potrzeb� rozszerzenia zbioru treningowego o sceny nocne i deszczowe oraz rozwa�enia zastosowania metod poprawy kontrastu i redukcji szumu w obrazie przed uruchomieniem detektora.

\subsection{Wybrane funkcjonalno�ci oprogramowania w eksperymencie offline}

Do realizacji eksperymentu offline wykorzystano kilka skrypt�w pomocniczych, odpowiedzialnych za kolejne etapy przygotowania danych, detekcji oraz ewaluacji. Poni�ej przedstawiono najwa�niejsze z nich wraz z przyk�adowymi fragmentami kodu.

\subsubsection{Skrypt \texttt{bounding\_boxes.py} -- generowanie anotacji referencyjnych}

Skrypt \texttt{bounding\_boxes.py}, dostarczony przez tw�rc�w symulatora CARLA, odpowiada za generowanie anotacji \textit{ground truth} w postaci ramek ograniczaj�cych w przestrzeni 3D i 2D. Dla ka�dego aktora obecnego w scenie wyznaczany jest obrys bry�y 3D, rzutowany nast�pnie na p�aszczyzn� obrazu kamery i zapisywany w formacie JSON.

\begin{lstlisting}[style=pythonColor,caption={Dodawanie obiektu do anotacji w \texttt{bounding\_boxes.py}},label={lst:bb_append}]
	json_frame_data['objects'].append({
		'id': npc.id,
		'class': SEMANTIC_MAP[npc.semantic_tags[0]][0],
		'blueprint_id': npc.type_id,
		'velocity': calculate_relative_velocity(npc, ego_vehicle),
		'bbox_3d': npc_bbox_3d['bbox_3d'],
		'bbox_2d': {
			'xmin': int(npc_bbox_2d['bbox_2d'][0]),
			'ymin': int(npc_bbox_2d['bbox_2d'][1]),
			'xmax': int(npc_bbox_2d['bbox_2d'][2]),
			'ymax': int(npc_bbox_2d['bbox_2d'][3]),
		} if npc_bbox_2d else None,
		'light_state': vehicle_light_state_to_dict(npc)
	})
\end{lstlisting}

Ka�dy wpis w strukturze \texttt{json\_frame\_data['objects']} zawiera identyfikator aktora, jego klas� semantyczn�, pr�dko�� wzgl�dn� wzgl�dem pojazdu ego, parametry bry�y 3D oraz wsp�rz�dne prostok�ta 2D opisuj�cego po�o�enie obiektu w obrazie. Tak przygotowane anotacje stanowi� punkt odniesienia w eksperymencie offline, s�u��c do por�wnania z detekcjami uzyskanymi z modelu YOLOv4.

\subsubsection{Skrypt \texttt{yolo.py} -- detekcja YOLO i zapis JSON}

Skrypt \texttt{yolo.py} realizuje detekcj� obiekt�w na zapisanych wcze�niej obrazach z kamery, korzystaj�c z biblioteki \texttt{ultralytics} i modelu YOLO. Wyniki detekcji zapisywane s� w plikach JSON w strukturze zbli�onej do formatu stosowanego przez \texttt{bounding\_boxes.py}, co u�atwia p�niejsz� ewaluacj�.

\begin{lstlisting}[style=pythonColor,caption={Struktura JSON z wynikami YOLO},label={lst:yolo_append}]
	for box in results.boxes:
	cls_id = int(box.cls[0])
	cls_name = model.names[cls_id]
	carla_class = yolo_to_carla_class(cls_name)
	xmin, ymin, xmax, ymax = box.xyxy[0].tolist()
	
	json_out["objects"].append({
		"id": obj_id,
		"class": carla_class,
		"blueprint_id": "yolo.detected",
		"velocity": {"x": 0, "y": 0, "z": 0},
		"bbox_3d": None,
		"bbox_2d": {
			"xmin": int(xmin),
			"ymin": int(ymin),
			"xmax": int(xmax),
			"ymax": int(ymax)
		},
		"light_state": {}
	})
\end{lstlisting}

Dla ka�dego wykrytego obiektu zapisywana jest klasa przemapowana na odpowiadaj�c� jej klas� w CARLA, prostok�t 2D w uk�adzie pikselowym oraz pomocnicze pola, takie jak identyfikator obiektu i informacje o stanie �wiate�. Ujednolicenie formatu danych umo�liwia bezpo�rednie zestawienie anotacji CARLA z detekcjami YOLO w kolejnym etapie eksperymentu.

\subsubsection{Skrypt \texttt{evaluate\_iou.py} -- obliczanie metryki IoU}

Skrypt \texttt{evaluate\_iou.py} odpowiada za ilo�ciow� ocen� jako�ci detekcji poprzez obliczenie warto�ci IoU mi�dzy ramkami \textit{ground truth} a ramkami przewidywanymi przez YOLOv4. Podstaw� jest funkcja wyznaczaj�ca Intersection over Union dla dw�ch prostok�t�w 2D.

\begin{lstlisting}[style=pythonColor,caption={Obliczanie IoU dla dw�ch ramek 2D},label={lst:iou_func}]
	def iou(boxA, boxB):
	xA = max(boxA["xmin"], boxB["xmin"])
	yA = max(boxA["ymin"], boxB["ymin"])
	xB = min(boxA["xmax"], boxB["xmax"])
	yB = min(boxA["ymax"], boxB["ymax"])
	
	inter = max(0, xB - xA) * max(0, yB - yA)
	areaA = (boxA["xmax"]-boxA["xmin"]) * (boxA["ymax"]-boxA["ymin"])
	areaB = (boxB["xmax"]-boxB["xmin"]) * (boxB["ymax"]-boxB["ymin"])
	union = areaA + areaB - inter
	return inter / union if union > 0 else 0.0
\end{lstlisting}

Na podstawie tej funkcji skrypt dopasowuje ramki YOLO do ramek referencyjnych, wyznacza �rednie IoU dla ka�dej klatki oraz globaln� �redni� IoU dla ca�ego przejazdu, a nast�pnie zapisuje wyniki do pliku CSV wykorzystywanego w analizie eksperymentu offline. Pozwala to na obiektywn� ocen� dok�adno�ci lokalizacji obiekt�w przy r�nych scenariuszach pogodowych i nat�eniu ruchu.

\subsubsection{Skrypt \texttt{bbox\_image.py} -- wizualizacja anotacji na obrazach}

Skrypt \texttt{bbox\_image.py} pe�ni rol� narz�dzia wizualizacyjnego, umo�liwiaj�cego na�o�enie ramek ograniczaj�cych zapisanych w plikach JSON na odpowiadaj�ce im obrazy. U�atwia to r�czn� inspekcj� jako�ci anotacji oraz tworzenie ilustracji przedstawiaj�cych przyk�ady dobrych, �rednich i s�abych dopasowa� ramek.

\begin{lstlisting}[style=pythonColor,caption={Rysowanie ramek na obrazie na podstawie pliku JSON},label={lst:bbox_image_short}]
	for obj in objects:
	bbox = obj.get("bbox_2d")
	if bbox is None:
	continue
	
	xmin = int(bbox["xmin"]); ymin = int(bbox["ymin"])
	xmax = int(bbox["xmax"]); ymax = int(bbox["ymax"])
	label = obj.get("class", "object")
	
	cv2.rectangle(img, (xmin, ymin), (xmax, ymax), (0, 255, 0), 2)
	cv2.putText(img, label, (xmin, ymin - 5),
	cv2.FONT_HERSHEY_SIMPLEX, 0.5, (0, 255, 0), 1)
\end{lstlisting}

Dla ka�dego obiektu odczytywanego z pliku JSON skrypt rysuje prostok�t 2D oraz podpis z nazw� klasy, a wynikowy obraz zapisywany jest do osobnego katalogu. Tak przygotowane wizualizacje wykorzystano w pracy do zilustrowania jako�ci detekcji w wybranych scenach eksperymentu offline.

\chapter{Dyskusja rezultat�w i wnioski ko�cowe}
%=================================================================================================
Celem niniejszej pracy by�o opracowanie systemu wykrywania i rozpoznawania obiekt�w takich jak znaki drogowe, pojazdy oraz piesi na obrazach pochodz�cych z kamery samochodowej, z wykorzystaniem algorytmu detekcji YOLO oraz �rodowiska symulacyjnego CARLA. Projekt zak�ada� implementacj� rozwi�zania w j�zyku Python, integracj� modelu g��bokiego uczenia ze �rodowiskiem symulacyjnym oraz weryfikacj� skuteczno�ci rozwi�zania poprzez testy wirtualne w r�nych warunkach drogowych.

Wszystkie cele okre�lone w zakresie pracy zosta�y zrealizowane:

\begin{itemize}
	\item Zapoznano si� z funkcjonalno�ci� �rodowiska CARLA, analizuj�c jego architektur�, mo�liwo�ci sterowania pojazdem oraz integracji z zewn�trznym oprogramowaniem.
	\item Przeprowadzono przegl�d metod wykrywania obiekt�w na obrazach z kamer samochodowych, ze szczeg�lnym uwzgl�dnieniem algorytmu YOLO jako rozwi�zania kompromisowego mi�dzy szybko�ci� a dok�adno�ci�.
	\item Zaimplementowano system detekcji obiekt�w, integruj�c YOLOv4 z symulatorem CARLA poprzez modyfikacj� skryptu \texttt{manual\_control.py}.
	\item Przeprowadzono testy, kt�re pozwoli�y oceni� skuteczno�� detekcji w r�nych scenariuszach symulacyjnych (np. zmienne warunki pogodowe, r�na liczba obiekt�w, perspektywa kamery).
	\item Sformu�owano wnioski oraz wskazano kierunki dalszego rozwoju systemu.
\end{itemize}

\subsection*{Wnioski}
Integracja detektor�w obiekt�w z symulatorami stanowi obecnie nieod��czny element testowania algorytm�w autonomicznej jazdy � zar�wno na etapie prototypowania, jak i walidacji modeli w �rodowiskach kontrolowanych.

Przeprowadzone eksperymenty wykaza�y, �e model YOLOv4 bardzo dobrze sprawdza si� w zadaniu wykrywania obiekt�w w czasie rzeczywistym. Uzyskane rezultaty by�y zadowalaj�ce zar�wno pod wzgl�dem liczby wykrytych obiekt�w, jak i ich klasyfikacji. Model skutecznie identyfikowa� pojazdy, ludzi oraz znaki drogowe w r�nych warunkach o�wietleniowych i pogodowych.

Na tle innych podej��, np. SSD czy Faster R-CNN, YOLOv4 wyr�nia si� znacznie wy�sz� pr�dko�ci� dzia�ania, co ma kluczowe znaczenie w kontek�cie pojazd�w autonomicznych. Cho� dok�adno�� detekcji mo�e by� nieco ni�sza ni� w przypadku modeli bardziej z�o�onych, to kompromis pomi�dzy szybko�ci� a precyzj� zosta� zachowany na bardzo dobrym poziomie, co potwierdza literatura przedmiotu oraz wyniki innych badaczy w tej dziedzinie.

Na podstawie przeprowadzonych test�w mo�na sformu�owa� nast�puj�ce wnioski:

\begin{enumerate}
	\item �rodowisko CARLA umo�liwia realistyczn� symulacj� warunk�w jazdy, co pozwala na skuteczne testowanie algorytm�w percepcyjnych bez konieczno�ci korzystania z rzeczywistych pojazd�w.
	
	\item Algorytm YOLOv4 zapewnia wysok� wydajno�� w czasie rzeczywistym, dzi�ki czemu mo�liwe jest wykrywanie wielu obiekt�w (samochody, piesi, znaki) przy zachowaniu p�ynno�ci dzia�ania systemu.
	
	\item Integracja detektora z kodem symulacyjnym CARLA, w tym przechwytywanie obrazu, przetwarzanie klatek oraz renderowanie detekcji, mo�e by� zrealizowana w spos�b stabilny i modularny. Modu�owa struktura umo�liwia �atwe rozszerzanie funkcjonalno�ci w przysz�o�ci.
	
	\item Wprowadzenie mo�liwo�ci filtrowania klas oraz dynamicznego renderowania wynik�w na ekranie znacz�co poprawia czytelno�� systemu i u�atwia analiz� zachowania modelu.
	
	\item Otrzymane rezultaty w pe�ni uzasadniaj� osi�gni�cie za�o�onych cel�w, gdy� zaprojektowany system wykrywa obiekty z du�� dok�adno�ci� i niskim op�nieniem, co odpowiada wymaganiom stawianym przed systemami percepcji w pojazdach autonomicznych.
\end{enumerate}

Mimo pozytywnych rezultat�w, nale�y wskaza� kilka ogranicze� i mo�liwo�ci ich eliminacji:

\begin{itemize}
	\item Model YOLOv4 nie zawsze poprawnie rozpoznaje obiekty w du�ym oddaleniu lub w s�abych warunkach o�wietleniowych, co mo�e wynika� z ogranicze� danych treningowych lub niskiej rozdzielczo�ci obrazu.
	
	\item Detekcja odbywa si� wy��cznie na podstawie danych wizyjnych � system m�g�by zosta� ulepszony poprzez fuzj� danych z innych sensor�w (np. LIDAR, radar), co poprawi�oby jego odporno�� na b��dy w trudnych warunkach.
	
	\item Wyniki testowanego systemu zosta�y przedstawione na podstawie zrzut�w ekranu, przedstawiaj�c pojedyncz� klatk�. Jako, i� system umo�liwia rozpoznawanie obiekt�w w czasie rzeczywistym nie by�o mo�liwe przedstawienie tego w formie pisemnej. 
	
	\item Testy przeprowadzono wy��cznie w �rodowisku symulacyjnym, kt�re � mimo wysokiego realizmu � nie oddaje w pe�ni nieprzewidywalno�ci �wiata rzeczywistego. Wdro�enie systemu na realnych danych wymaga�oby dalszego dostrojenia i oceny.
	
	\item Z powodu ogranicze� sprz�towych system zosta� przetestowany w nieco gorszych warunkach, ni� zalecanych przez sam symulator CARLA. Dodatkowo system do detekcji obiekt�w YOLO pobiera kolejne zasoby sprz�towy na skutek czego w gorszych warunkach pogodowych, takich jak noc czy deszcz oraz przy du�ym nat�eniu ruchu ilo�� klatek by�a bardzo niska.
\end{itemize}

\subsection*{Mo�liwy rozw�j systemu}
W oparciu o zdobyt� wiedz�, mo�liwe s� nast�puj�ce kierunki rozwoju systemu:

\begin{itemize}
	\item Zastosowanie modelu YOLOv8 lub innych nowszych architektur, kt�re oferuj� lepsz� dok�adno�� i mo�liwo�� pracy na mniejszych urz�dzeniach (np. Jetson Nano, Raspberry Pi).
	
	\item Rozszerzenie systemu o modu� decyzyjny, kt�ry na podstawie wykrytych obiekt�w podejmuje dzia�ania � np. zatrzymanie pojazdu, zmiana pasa, ostrze�enie o niebezpiecze�stwie.
	
	\item Wprowadzenie mechanizm�w oceny jako�ci detekcji (precision, recall, mAP) oraz zapisywanie wynik�w do analizy por�wnawczej.
	
	\item Implementacja systemu �ledzenia obiekt�w (np. Deep SORT), co pozwoli�oby na analiz� trajektorii oraz zachowa� uczestnik�w ruchu drogowego.
	
	\item Testowanie systemu na rzeczywistych nagraniach z kamer samochodowych, celem oceny jego przydatno�ci w praktycznych wdro�eniach.
\end{itemize}

Ostatecznie, praca w pe�ni zrealizowa�a zak�adane cele i potwierdzi�a skuteczno�� po��czenia narz�dzi symulacyjnych z algorytmami g��bokiego uczenia w kontek�cie system�w percepcyjnych. W oparciu o przeprowadzone badania oraz analiz� wynik�w mo�na stwierdzi�, �e integracja YOLOv4 z symulatorem CARLA stanowi efektywne, elastyczne i przysz�o�ciowe narz�dzie do projektowania system�w rozpoznawania otoczenia w pojazdach autonomicznych.



%========================================================================================
% Literatura
%========================================================================================

\begin{flushleft}
	\bibliography{bibliografia/bibliografia}
\end{flushleft}

\end{document}
%========================================================================================
% Koniec dokumentu
%========================================================================================
